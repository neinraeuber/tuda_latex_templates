\begin{itemize}
	\item sinvvol, das hier schon zu erwähnen ?? oder erst im Messaufbau??? --> ist es sonst dopplung?
	\item vllt. hier nur Überblick, im Aufbau-Teil dann genauer die einzelnen Komponenten erklären.
\end{itemize}

Spectroscopy measures absorbtion and emission of light and other radiation by matter \cite{atascientificUnderstandingSpectrometrySpectroscopy2020}. The energy, wavelength, or frequency of photons that pass through a sample is detected that way. In the particular case of THz-TDS, matter is probed with very short Thz pulses.
The pulse usually has a duration of a few picoseconds \cite{neuTutorialIntroductionTerahertz2018}. The THz time domain signal measures the trasient electrical field. This is a major advantage of THz-TDS because intensity and phase of the electrical field are measured simultaneously \cite{zhaoPrincipleTerahertzTimeDomain2023}.

A conventional THz-TDS system consists of a femtosecond laser, a terahertz source, mirrors for beam steering, delay stages, optical beam splitters, focusing and collimating optics such as parabolic mirrors, and a detector. The working principle of the most important individual components is described in the following. 

\textbf{TODO: sinnvoll, so kleinschrittig vorzugehen??}
\textbf{Abbildung von THz Puls und Fourier Transformierter}


\subsection{Femtosecond Laser}
In a pulsed THz-TDS system, an unltrafast laser is used to drive both the THz detector and source. The laser's pulse duration lies in the $\sim$ \num{100} \si{\femto\s} range. 

\subsection{Terahertz Sources and Detector}

\begin{itemize}
	\item welchen laser benutzen wir????
\end{itemize}

\begin{itemize}
	\item Emission, Detection using PCAs 
	\item explain important components (Laser, Delay Stage etc.)
\end{itemize}