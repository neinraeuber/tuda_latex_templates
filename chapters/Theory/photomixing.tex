\begin{itemize}
	\item irgendwie darauf eingehen, warum nur Slotline und H-Dipole in Frage kommen
	--> irgendwas mit Dispersion und so (machen Frequenzen würden zB bei LogSpiral länger durchlaufen, andere kürzer) 
\end{itemize}
\begin{itemize}
	\item Slotline vs. H-Dipole; talk about how it is basically only
	applicable to use those two 
\end{itemize}

Photoconductive antennas (PCAs) are used to emmit THz pulses. Etc. pp. 

Photoconductors are semiconductor devices with a direct band-gap. These devices are operated with lasers and are based on the principle of photomixing. Two interfering laser beams forming an optical beat signal are focused on the photoconductor. When the beat signals energy level is higher than that of the semiconductors band-gap energy, electron-hole pairs are photo-generated. A bias voltage applied on electrodes of the device separates the carriers to create a displacement current. The current oscillates with the envelope of the absorbed laser light. It results in a THz signal \cite{nandiErAsInAlGaAsPhotoconductors2021}. 

The Thz signal is emitted or detected by photoconductive antennas (PCAs). PCAs usually consist of a photomixing device sitting in between the antenna's electrodes. The electrodes help emmit or detect the THz signal. The mathematics for THz generation through photomixing are given in the following sections.


\subsection{Principles of Photomixing for pulsed Operation}
To explain pulsed operation for THz-TDS, continuos wave (CW) opertaion (see figure \textbf{ABB einfügen}) is explained beforehand for easier comprehension. This chapter is mainly a review of \cite{nandiErAsInAlGaAsPhotoconductors2021,faridiPulsedFreeSpace2023,preuPrinciplesTHzGeneration2015}.

\subsubsection{CW operation}
CW THz radiation is generated using two CW lasers. The lasers have a difference frequency of $\omega_L = \omega_0 \pm \omega_{THz}$. The photon energy of the lasers has to be higher the band-gap energy of the photoconductor material: $h\nu_L > E_G$, 
where $\nu_L = \omega_L / 2\pi$, $h$ is Planck’s constant and $E_G$ is the bandgap energy of the material. The two laser signals are heterodyned (frequency mixed) using a fiber coupler before they are fed into the photoconductor with an optical field of

\begin{equation}
	\vec{E}(t) = \vec{E}_1(t) + \vec{E}_2(t) = \vec{E}_{1,0}e^{i(\omega_L - \omega_{THz}/2)t} + \vec{E}_{2,0}e^{i(\omega_L + \omega_{THz}/2)t - i\phi},
\end{equation}
where $\phi$ is the phase difference of the laser signals. The optical intensity of the absorbed laser power is given by 
\begin{equation}
	I_L(t) \sim |\vec{E}(t)|^2 = |\vec{E}_{1,0}(t)|^2 + |\vec{E}_{2,0}(t)|^2 + 2|\vec{E}_{1,0}(t) \cdot \vec{E}_{2,0}(t)|\cos(\omega_{THz}t + \phi), 
\end{equation}

which can be rewritten in terms of laser power as 

\begin{equation}
	P_L(t) = P_1 + P_2 + 2\sqrt{P_1 P_2}\cos(\beta)\cos(\omega_{THz}t + \phi), 
\end{equation}

where $\beta$ is the angle between the polarization of the electric fields. In an ideal photoconductor all laser light is absorbed. This yields in an ideal photocurrent 
\begin{equation}
	I_{Ph}^{Id}(t) = \frac{eP_L(t)}{h\nu_L} = \frac{e(P_1+P_2)}{h\nu_L} + 2\frac{e\sqrt{P_1P_2}}{h\nu_L}\cos(\beta)\cdot\cos(\omega_{THz}t + \phi).
	\label{eq_iph}
\end{equation}

We see that the ideal photocurrent $I_{Ph}^{Id}(t)$ consists of a DC component $I_{DC}^{Id}$ and an AC component $I_{THz}^{Id}$.
The DC component is given by 
\begin{equation}
	I_{DC}^{Id} = \frac{e(P_1+P_2)}{h\nu_L}.
\end{equation} 
The AC component of the ideal photocurrent is given by the expression
\begin{equation}
	I_{THz}^{Id}(t) = 2\frac{e\sqrt{P_1P_2}}{h\nu_L}\cos(\beta)\cdot\cos(\omega_{THz}t + \phi).
\end{equation}

To maximize the THz output, the AC part of the photocurrent has to be maximized. We see that $I_{THz}^{Id}$ is at its maximum when $P_1 = P_2 = P_L = P_{tot} / 2$ and $\beta = 0$. This is equivalent to identical power and polarization of the two laser signals. With these assupmtions the amplitude of the ideal photocurrent's THz part becomes $I_{THz}^{Id} = eP_{tot} / (h\nu) = I_{DC}^{Id} = I^{Id}$. Ultimately inserting our assupmtions into \ref{eq_iph} we get 
\begin{equation}
	I_{Ph}^{Id} = I^{Id}[1 + \cos(\omega_{THz}t + \phi)].
\end{equation}

The generated photocurrent is usually fed into an antenna. This antenna is fabricated on the same semiconducting material as the photomixing device. The antenna shows a radiation resistance $R_A$, yielding in the ideal emitted THz power 
\begin{equation}
	P_{THz}^{Id}=\frac{1}{2}R_A (I_{THz}^{Id})^2.
\end{equation}

\subsubsection{Pulsed Operation}

For pulsed operation, the semiconductor absorbs a short optical pulse. The photocurrent generated by this pulse results in THz radiation when fed into an antenna e.g. The emitted THz field is proportional to the time derivative of the photocurrent generated by the lasers \cite{preuTunableContinuouswaveTerahertz2011}.
\begin{equation}
	E_{THz} \propto \frac{\partial I_{Ph}}{\partial t}.
\end{equation}

The CW generation of THz radiation can be extended to the pulsed approach. Under pulsed operation multiple laser frequencies are mixed to form a single laser pulse denoted by 
\begin{equation}
	\vec{E}(t) = \sum_j^n \vec{E}_je^{i(\omega_j t + \phi j)}.
\end{equation}
Here, $\omega_j = 2 \pi \nu_j$ are the angular frequency components of each electrical field $\vec{E}_j$ and $\phi_j$ are the corresponding phase components. Assuming a typical mode locked femtosecond laser we get the following properties: 
\renewcommand{\labelenumi}{\alph{enumi})}
\begin{enumerate}
	\item The frequency components are equally spaced in the spectral domain: $\nu_j - \nu_{j-1} = R_p$. $R_p$ is the repetition rate of the laser.
	\item The phase has a fixed relationship that is linear: $\phi_j - \phi_{j-1} = const.$
\end{enumerate}

The mode locking allows for very short laser pulses in the femtosecond range.

\textbf{TODO: muss mal noch gucken was hier an theory sonst noch wichtig ist. irgendwas mit fourier auf jeden fall.}



\begin{itemize}
	\item beginnen mit CW Operation weil pulsed Operation quasi davon abgeleitet ist???
	\item dann quasi uttam nandi thesis so übernehmen ??
\end{itemize}

\subsection{Photoconductive Antennas for THz-TDS}
