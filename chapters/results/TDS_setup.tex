A typical THz-TDS setup  (see Figure \ref{fig_TDS}) is presented in the following. The post-processing of the obtained data from THz-TDS measurements is explained. 

\subsection{THz-TDS Measurement Setup}

\begin{figure}[!b]
    \includegraphics[width=0.9\linewidth]{figures/TDS_schematic.pdf}
    \centering
    \caption{Schematic diagram of the setup used for THz-TDS.}
    \label{fig_TDS}
\end{figure}

Generation and detection of THz radiation in a pulsed system were described in detail in Section \ref{sec_gen_det}. Only a brief overview will be given here. The TSD setup used for characterization of the fabricated H-Dipole antennas utilizes a modified MenloSystems C-Fiber \cite{FemtosecondErbiumLaser} laser system with a pulse duration of \num{90} \si{\femto \s}, a repetition rate of \num{100} \si{\mega \hertz} and an optical wavelength of \num{1560} \si{\nano \meter}. The laser signal is coupled out using glass fibers. In order to achieve coherent signal detection, the THz-TDS setup utilizes a closed-loop system. Both the pump beam and the probe beam necessary for THz generation and detection with PCAs are generated using the same laser beam. The fibers drive both the detector and the emitter with \num{25} \si{\milli \watt} of power each. The THz signal is generated with the help of the pump beam and ultimately detected by sampling the temporal overlap of the probe beam and the THz field. This closed-loop approach ensures that the THz-TDS system provides high SNR and DNR. A beam splitter is utilized to split the initial laser pulse into the two needed signals. 

On the transmission side of the setup, one of the two laser signals passes through a delay stage before being fed into the transmitter (Tx) developed in \cite{nandiErAsInAlGaAsPhotoconductors2021}. The transmitter primarily consists of a PCA mounted on a hyper-hemispherical silicon lens with a radius of \num{6.1} \si{\milli \meter}. The device is fully packaged, featuring an APC-fiber for the laser signal and a SMA connector for applying the bias voltage. The package has a cylindrical shape with a length of \num{5} \si{\centi \meter} and an outer diameter of \num{6} \si{\centi \meter}. Additionally, the packaging includes a fiber collimator and a focusing lens to ensure efficient coupling of the laser signal into the device. The PCA is DC-biased until a current of approximately $180$ \si{\micro \ampere} is reached, enabling the acceleration of free charge carriers generated by the laser signal. This typically corresponds to a bias voltage of around \num{180} \si{\volt}. The fully packaged transmitter is mounted on a one-dimensional stage to horizontally align the THz field. When illuminated by the laser signal, the PCA emits THz radiation. The THz field travels through free space and is focused onto the receiving port by two lenses.  

As demonstrated in Section \ref{sec_gen_det}, the signal at the receiver in this setup can be approximated as the convolution of the delta-function and the incident THz transient. Hence, a mechanical delay stage is needed before the laser signal reaches the transmitter to scan through the complete THz pulse. By varying the relative delay between the THz signal and the optical probing pulse, the complete THz pulse received by the antenna can be extracted. 

The second laser signal is directly coupled to the receiving port (Rx) of the setup. The receiver consists of an antenna that is being used to work as a recipient for THz radiation.  In this thesis, the receiving port of the setup is an antenna of the fabricated probe, which serves as the device under test (DUT). The DUT is mounted on a hyper-hemispherical silicon lens with a radius of \num{6.1} \si{\milli \meter}, using vacuum grease for attachment. A fiber collimator and a focusing lens are employed to direct the laser pulse onto the active region of the DUT. Two three-dimensional stages are used: one for aligning the laser signal with the DUT, and another for aligning the DUT with the incident THz transient.

For a PCA to work as a receiver, the device is illuminated by the femtosecond laser pulse, the probing pulse. By illuminating the device, free electron-hole pairs are generated. The incoming THz radiation, which is to be measured by the receiving port, biases the device. The illumination and biasing of the device results in a DC photocurrent. This current is proportional to the convolution of the incoming THz transient and the optical probing signal. The DC current can be read out using probing needles at the antenna pads and post detection electronics. As the measured photocurrent is generally very small (around $10^{-9} ... 10^{-6}$ \si{\ampere}), the signal is transformed to a voltage and amplified by a trans-impedance amplifier (TIA) from TEM-Messtechnik (PDA-S) \cite{PDASPhotodiodenVerstaerker}. With the TIA, gains of up to $\sim 10^7 $ \si{\volt}/\si{\ampere} can be achieved. The THz signal needs to be extracted from background noise. A Lock-In Amplifier (LIA) is used for signal extraction. The input signal is multiplied with a reference signal (demodulation), which is usually a sine-wave, and then filters the result using a low-pass filter. This isolates the desired signal at a specific frequency from noise and other frequency components. The frequency at which the demodulation takes place is the same frequency used for modulating the THz transmitter. A LIA from Zurich Instruments (MFLI-MD) \cite{MFLI500KHz2019} is employed for demodulation of the incoming signal. A graphical user interface provided by MenloSystems is used for data acquisition and storage. For all measurements, a scanning window of \num{100} \si{\pico \s} is employed. The data is recorded with an integration time of \num{1000} \si{\s} to achieve the highest possible SNR. The time-domain data is saved for further data analysis.


\subsection{Data Processing and Analysis}

The measurement data is further processed in MATLAB to get a clearer picture of the parameters we are interested in. The data processing mainly concerns the evaluation of the THz performance of the antennas. In order to gain knowledge about the THz performance, we analyze the time-domain oscillations caused by non-radiating reflections within the feeding strip. The time-domain data is temporally aligned to facilitate comparison of these oscillations. A digital low-pass filter, namely a moving average (MA) filter, is applied to the time-domain traces. The signals are normalized to their respective amplitudes. Averaging and normalizing the data is particularly helpful for noisy signals, as it allows the oscillations caused by strip reflections to be extracted from the noise. These post-processing steps may distort the absolute amplitude information, potentially exaggerating oscillation amplitudes compared to raw measurements. However, this is acceptable for this work, since our primary focus is on the relative influence of NiCr on the oscillations compared to the reference antennas. As we are primarily interested in low-frequency time-harmonics ($\nu < 500$\,\si{\giga \hertz}), the window size for calculating the MA is chosen to have its cutoff frequency $\nu_c$ at \num{500}\,\si{\giga \hertz}. The frequency response of a MA filter is given by the DFT of a rectangular pulse, yielding 
\begin{equation}
    |H(\nu)| = \frac{\sin (\pi \nu N \Delta t)}{N\sin (\pi \nu \Delta t)}.
    \label{eq_MA}
\end{equation}

Here, $N$ denotes the window size for the MA filter and $\Delta t$ is the step size in our time-domain measurements. Using $|H(\nu_c)| = 1/\sqrt{2}$ and thus defining the cutoff frequency as the \num{3}\,\si{\decibel} point of the frequency response, eq. \eqref{eq_MA} becomes
\begin{equation}
    \frac{1}{\sqrt{2}} = \frac{\sin (\pi \nu_c N \Delta t)}{N\sin (\pi \nu_c \Delta t)}. 
    \label{eq_MA_2}
\end{equation}

The step size in our measurements is $\Delta t \approx. 35$\,\si{\femto \s}. With a desired cutoff frequency of $\nu_c = 500$\,\si{\giga \hertz}, the window size is set to $N = 25$, as eq. \eqref{eq_MA_2} becomes 
\begin{equation}
    \frac{\sin (25 \cdot \pi \cdot 500\,\si{\giga \hertz} \cdot 35\,\si{\femto \s}) }{25 \cdot \sin (\pi \cdot 500\,\si{\giga \hertz} \cdot 35\,\si{\femto \s})} \approx 0.71 \approx \frac{1}{\sqrt{2}}
\end{equation}

The spectra are obtained by calculating the FFT of a measured time-domain trace. To compare performance factors such as bandwidth and DNR, the individual spectra are normalized to their respective noise levels. The noise level of a signal is determined by averaging the spectrum over the range from \num{4}\,\si{\tera \hertz} to \num{6}\,\si{\tera \hertz}. This range is utilized for noise floor calculations as no device was able to receive signal above the \num{4}\,\si{\tera \hertz} threshold.  A MA filter is applied to the normalized signal to facilitate evaluation of the bandwidth. Averaging the spectrum improves differentiation between actual signal and noise, as the frequency-domain data gets smoothed. The window size of the MA filter is chosen at $N=10$. The spectral bin step is $\Delta \nu \approx 7$\,\si{\giga \hertz}, resulting in a smoothed span of \num{70}\,\si{\giga \hertz}. The magnitude of the spectra is displayed on a logarithmic scale. Since the received THz power is proportional to the square of the measured photocurrent, the spectra are calculated as follows: $P_{dB} = 20\log_{10}(I_{Ph})$. By normalizing the spectrum to its noise floor, we effectively calculate the DNR of the signal yielding $P_{DNR} = 20\log_{10}(I_{Ph}/I_{noise})$. Bandwidth in this work is then defined as received THz power above the noise floor, yielding $P_{DNR} > 0$\,\si{\decibel}.   

