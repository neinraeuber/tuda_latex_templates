\section{Summary}

\noindent
The objective of this thesis was to investigate strategies for suppressing parasitic low-frequency resonances in
photoconductive antennas (PCAs), which are a limiting factor in terahertz time-domain spectroscopy (THz-TDS).
These resonances, originating from the metallic contact pads and finite feeding strips, manifest as oscillations
in the time-domain and sharp spectral peaks around 50–100\,GHz, thereby masking broadband information
and reducing the usable bandwidth of PCA-based systems.

\noindent
A detailed analysis of the resonance mechanism was carried out through simulations of a reference H-Dipole
geometry. It was shown that the large read-out pads and feeding strips act as resonant structures that couple to
long-wavelength currents. To mitigate these effects, resistive NiCr segments were introduced into the feeding
lines. Simulations demonstrated that these segments effectively damp the unwanted low-frequency currents,
while leaving the high-frequency response around 1\,THz essentially unaffected. Radiation impedance analysis
further confirmed that NiCr-modified antennas exhibit a flatter and broader spectral response compared to the
reference design.

\noindent
Based on these designs, a series of H-Dipole and I-shaped Dipole antennas with varying NiCr lengths was
fabricated. The structures were characterized by THz-TDS. The experimental results validated the simulation
predictions: antennas with longer NiCr sections showed a marked reduction of parasitic oscillations in the
time-domain as well as smoother frequency spectra. A trade-off between resonance suppression and absolute
signal amplitude was observed, reflecting the resistive damping mechanism. Despite fabrication tolerances
such as lithography deviations and variations in NiCr sheet resistance, the overall agreement between
simulation and measurement confirms the effectiveness of the proposed approach.

\section{Outlook}

\noindent
The results of this work demonstrate that the introduction of resistive NiCr segments provides a simple yet
powerful means of suppressing parasitic pad resonances in photoconductive antennas. At the same time, the
study highlights several aspects that should be addressed in future research:

\begin{itemize}
    \item \textbf{Optimization of resistance:} A systematic parametric study of NiCr thickness, sheet resistance,
    and segment length could identify the optimal trade-off between resonance suppression and signal amplitude.
    
    \item \textbf{Alternative antenna geometries:} The concept should be extended to broadband designs such as
    bow-tie, log-periodic, or spiral antennas to assess its general applicability.
    
    \item \textbf{Advanced materials:} Other resistive materials (e.g. TiN, amorphous carbon, graphene) may
    offer improved tunability, lower parasitic losses, or more stable fabrication processes compared to NiCr.
    
    \item \textbf{Integration with device design:} Combining resistive feeds with other mitigation strategies, such
    as optimized pad geometry or dielectric matching layers, could yield further performance gains.
    
    \item \textbf{Application perspective:} Improved PCA bandwidth directly benefits THz spectroscopy and
    imaging applications, particularly in areas where both low- and high-frequency components are critical,
    such as material characterization, non-destructive testing, and high-speed wireless communications.
\end{itemize}

\noindent
In summary, this thesis shows that resistive NiCr feeding segments are a viable approach for eliminating
parasitic resonances in photoconductive antennas. The findings provide both a conceptual foundation and
experimental validation, offering a pathway towards more robust, broadband, and application-ready THz
systems.
