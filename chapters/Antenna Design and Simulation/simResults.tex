The CST simulations provide insight into how resistive NiCr segments affect surface current distributions and
radiation impedances. The distribution of the surface current is a measure of the electromagnetic wave's propagation through the antenna structure.  We present the surface current distribution in four H-Dipole antennas with differing NiCr-sections. The four configurations are compared at two distinct frequencies: \num{100} \si{\giga \hertz} (see Figure \ref{sc_100ghz_comp}) and \num{1} \si{\tera \hertz} (see Figure \ref{sc_1thz_comp}). The radiation impedance is also analyzed, giving insight on resonant peaks, as well as broadband performance.

\subsection{Surface Current Distribution at 100 GHz}

\begin{figure}[!]
    \centering
    \includegraphics[width=\linewidth]{figures/Contour_Plots_v2/100Ghz_SC_sim_plots.pdf}
    \caption{Contour plots of the simulated surface currents at \num{100} \si{\giga \hertz} for four different H-Dipole antenna configurations. The surface current is plotted logarithmically. a) Surface current distribution in the reference H-Dipole. b) Surface current distribution with the NiCr-section corresponding to a resistance of $R_{NiCr} = 300$ \si{\ohm}. c) Surface current distribution with the NiCr-section corresponding to a resistance of $R_{NiCr} = 1000$ \si{\ohm}. d) Surface current distribution with the NiCr-section corresponding to a resistance of $R_{NiCr} = 3740$ \si{\ohm}.}
    \label{sc_100ghz_comp}
\end{figure}
 
 The simulations at \num{100} \si{\giga \hertz} show that adding a NiCr-section to the antenna feed should reduce the low frequency surface current spikes caused by the antenna feeds and pads.
Figure \ref{sc_100ghz_comp} a) depicts the surface current in the reference H-Dipole antenna where no NiCr is added. A fairly even distribution of the surface current along the antenna structure can be observed. Red sections along the antenna's feeding strip indicate that a high extend of the THz radiation is emitted in the strip, indicating the leaky-wave behavior of the antenna at low frequencies. Leaky-wave behavior refers to the phenomenon where a guided electromagnetic wave gradually radiates energy into the surrounding medium as it propagates along a structure. Low-frequency harmonics are caused by  resonances along the feeding strip. We also see that the surface current distribution in the electrodes and the pads is fairly similar at surface currents of a few \num{100} \si{\ampere/\meter}, causing the low frequency resonances. 

A NiCr-section equivalent to a resistance of $R_{NiCr} = 300$ \si{\ohm} is added to the reference H-Dipole in Figure \ref{sc_100ghz_comp} b). Already, we can observe a concentration of the surface current around the antenna's electrodes. The surface current distribution in the pads now only reaches values of a few \num{10} \si{\ampere/\meter}. Low frequency pad resonances still occur but with a damped amplitude. Along the feeding strips, resonant spots are observable, indicating that low frequency harmonics are still present. 

At $R_{NiCr} = 1000$ \si{\ohm} (see Figure \ref{sc_100ghz_comp} c), the surface current is heavily concentrated around the antenna's electrodes. The surface current distribution in the pads nearly drops to zero. The surface current along the feeding strip appears relatively homogenous, indicating an absence of leaky-wave modes along the axis parallel to the strip and thus a reduction of low frequency time-harmonics. This is the non-resonant low frequency behavior we want to achieve. $R_{NiCr} = 3740$ \si{\ohm} (see Figure \ref{sc_100ghz_comp} d)) is the maximum resistance we can achieve with the chosen simulation parameters of a maximum strip length of $l_{strip} = 1900$ \si{\micro \meter} and a NiCr sheet resistance of \num{22} \si{\ohm/sq}. A very similar surface current distribution compared to figure \ref{sc_100ghz_comp} c) is observable. The surface current along the feeding strip is further reduced and continues to concentrate around the electrodes. 

\subsection{Surface Current Distribution at 1 THz}

\begin{figure}[!]
    \centering
    \includegraphics[width=\linewidth]{figures/Contour_Plots_v2/1Thz_SC_sim_plots.pdf}
    \caption{Contour plots of the simulated surface currents at \num{1} \si{\tera \hertz} for four different H-Dipole antenna configurations. The surface current is plotted logarithmically. a) Surface current distribution in the reference H-Dipole. b) Surface current distribution with the NiCr-section corresponding to a resistance of $R_{NiCr} = 300$ \si{\ohm}. c) Surface current distribution with the NiCr-section corresponding to a resistance of $R_{NiCr} = 1000$ \si{\ohm}. d) Surface current distribution with the NiCr-section corresponding to a resistance of $R_{NiCr} = 3740$ \si{\ohm}.}
    \label{sc_1thz_comp}
\end{figure}

It is important that adding the NiCr-section only affects the surface current generated at lower frequencies. At higher frequencies ($\nu > 100$ \si{\giga \hertz}), the coupling of THz radiation into the antenna is expected to occur predominantly at the electrodes due to the decreasing wavelength. Ideally, the NiCr segments should exert minimal influence on the surface currents propagating from the electrodes to the antenna pads. Figure \ref{sc_1thz_comp} shows the simulated surface current distribution at \num{1} \si{\tera \hertz} in the four antenna configurations which were already discussed at \num{100} \si{\giga \hertz}. 

At \num{1} \si{\tera \hertz}, the surface current distribution appears largely invariant across different antenna configurations. This antenna behavior is intended. The incorporation of NiCr segments introduces a resistive component that attenuates surface currents at lower frequencies. At higher frequencies, the surface current is able to propagate despite the added resistance. 

\subsection{Radiation Impedance for Different NiCr Modifications}

Another way to evaluate the impact of incorporating NiCr into the antenna feeding strip is by analyzing the radiation impedance of the simulated antenna. In the reference antenna that features no NiCr, sharp resonances appear. The cause of these resonances was discussed in Section \ref{sec:padResonances}. Non-radiative lower frequencies are reflected as standing waves, causing resonances spaced apart by approx. \num{50}\,\si{\giga \hertz}. For the antennas modified with NiCr, we see that small amounts of NiCr corresponding to fairly low resistances already seem to suppress these low frequency oscillations. The first major resonance sustains however and is shifted to even lower frequencies. The actual effect of this resonance shift has to be observed in measurements. Overall, the NiCr-modified antennas show a flatter and non-resonant radiation impedance. The non-resonant curves result in higher effective radiation impedances between \num{0}\,\si{\giga \hertz} and \num{500}\,\si{\giga \hertz}. The average radiation impedance of the reference antenna across this frequency range is \num{137.88}\,\si{\ohm}. The average radiation impedance of the antenna containing NiCr segments of length \num{240}\,\si{\micro \meter} is \num{153.55}\,\si{\ohm}. 
At higher frequencies above \num{500}\,\si{\giga \hertz}, the modified antenna's and the reference antenna's radiation impedances converge, indicating that the high frequency behavior of the antennas is not impacted by the introduction of NiCr. 


\begin{figure}[!]
    \centering
    \includegraphics[width=0.85\textwidth]{figures/appdx/sim_rad_imp_H_Dipoles.pdf}
    \caption{Radiation impedance of H-Dipole antennas with the stated properties at different lengths for the NiCr strip corresponding to different resistances.}
    \label{}
\end{figure}

\subsection{Discussion of Simulation Results}

The simulations confirm the anticipated behavior.  Resistive NiCr feeds effectively suppress the low-frequency resonances caused by pads and feeding strips. At higher frequencies (e.g., \num{1}\,\si{\tera\hertz}), the surface current distribution shows that the electromagnetic wave propagation through the antenna structure remains essentially unchanged, indicating that the broadband performance of NiCr-modified antennas is not compromised. The higher mean radiation impedance observed for the modified antennas compared to the reference H-Dipole suggests a potentially reduced RC roll-off factor. The simulation results provide strong motivation for the fabrication and experimental validation of NiCr-modified antennas.