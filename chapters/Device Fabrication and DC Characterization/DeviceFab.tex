The device fabrication consists of three main processes. Included in these processes are various litography steps, evaporation of different metals, wet etching and deposition of silicon nitride as an anti-reflective coating. First, the desired antenna structure is imprinted on our wafer through contact litogrpahy and the antennas metal is deposited. Then, mesa litography and mesa etching is applied. In the end, an antireflective coating is deposited. 

\subsection{Antenna Structure Litography and Metal Deposition}

\textbf{TODO: ABBILDUNG}

A litography mask is fabricated containing the desired antenna structures. A \num{7} $\times$ \num{8} \si{\milli\meter} sample of the ErAs:In(Al)GaAs photoconductor, grown on a \num{500} \si{\micro\meter} InP:Fe substrate wafer is cleaved out from the wafer. The desired antenna structures are imprented onto the sample via contact litography. Structures as small as approximately \num{1} \si{\micro\meter} are desired, so good contact is critical in this process. 

A thin layer of AZ 5414E image reversal photoresist is deposited on the sample. By spin coating the sample with photoresist, a uniform layer is achieved. The sample is then developed using AZ MIF 726, leaving a \num{6} $\times$ \num{7} \si{\milli\meter} of photoresist. This helps in having good contact when applying contact litography steps.

The image reversal effect of AZ 5414E is achieved by baking the sample at \num{120} \si{\celsius} for one minute and exposing it to ultraviolet light of five times the initial exposure time. The AZ 5415E is transformed from positive to negative. A second litographic step is performed leaving the actual antenna structures exosed. 

Before depositing the metal for the PCAs, the samples are dipped into a 1:1 solution of H\textsubscript{2}O and HCl for \num{30} seconds to remove the oxide layer which forms at the surface. This step greatly improves the adhesion of the deposited metal. The antenna electrodes and contact pads are fabricated by depositing a \num{180} \si{\nano\meter} layer of gold (Au) on top of a \num{20} \si{\nano\meter} layer of Chrome (Cr) via electron beam evaporation. The Cr-layer improves the contact with the semiconducting material. The antenna feeds are fabricated by combining the deposition of an Au-layer on top of a Cr-layer and a nichrome (NiCr) layer. The two layers are joined.  

\textbf{TODO: maybe further explain how NiCr is deposited}

After completing the metal deposition step, the sample is dipped in acetone. This way we get rid of the unwanted metal on our sample. 

An additional annealing step is needed after metal deposition to further improve the metal-semiconductor-contact. Such an annealing process reduces the interface impurities and creates good contact on the atomic scale between the metal and the semiconducting material \cite{tahamtanInvestigationEffectAnnealing2011}. Annealing is performed at a temperature of \num{425} \si{\celsius} for \num{30} seconds. After annealing, the antennas IV-characteristics are measured to ensure ohmic contacts. 

\subsection{Mesa Litography and Mesa Etching}

A thick layer of AZ 1518 HS photoresist is deposited onto the sample. Again, spin scoating ensures a uniform photoresistive layer. This step is followed by hard baking the sample covered by photoresist at \num{110} \si{\celsius} for \num{15} minutes. The hard baking makes the residual solvent present in the resist evaporate, stabilizes the printed structure and improves the bond between the resist and the material. 

The thin ErAs:In(Al)GaAs photocondutive material is etched from the areas that were not protected by the photoresist layer using 
hydrogen peroxide (H\textsubscript{2}O\textsubscript{2}), sulphuric acid (H\textsubscript{2}SO\textsubscript{4}) and water mixed in appropriate proportion. The semi-insulating InP:Fe substrate is left exposed in those areas. 

By mesa etching, the active InGaAs layer is removed from the entire sample except between the electrodes and under the antenna and pads. This process ensures a high resistance is with a minimal dark current. A minimal dark current drastically improves the signal-to-noise ratio of the THz signal. 

\textbf{TODO: ABBILDUNG}

\subsection{Anti-Reflection Coating Deposition}
An anti-reflection coating (ARC) is deposited above the active region. The ARC layer helps improving the transmission of the THz by minimizing reflection \cite{chenAntireflectionImplementationsTerahertz2014} and protects the device from external damage. The ARC is fabricated by growing a silicon nitride (Si\textsubscript{3}N\textsubscript{4}) layer on top of the sample via plasma-enhanced chemical vapor deposition. 

After depositing the ARC, another thick layer of AZ 1518 HS photoresist is deposited on the sample to form the required shapes of the ARC over the mesa. The photoresist only covers the antennas electrode structure. 

The Si\textsubscript{3}N\textsubscript{4} is etched out from the unprotected areas, exposing the contacting metal pads attached to the antenna. The etching is done in a plasma etching machine using carbon tetra-fluoride (CF\textsubscript{4}) as the etching agent. After etching of Si\textsubscript{3}N\textsubscript{4} the photoresist is removed by dipping the sample into acetone.


\textbf{TODO: ABBILDUNG maybe}