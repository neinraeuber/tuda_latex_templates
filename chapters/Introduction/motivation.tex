With an increasing number of applications the, demand for efficient THz devices is steadily increasing. THz technology is rapidly advancing, enabling new applications in fields like healthcare, telecommunications, and security. As hardware improves, THz is poised to transform industries by enhancing data transmission, diagnostics, and integration with emerging technologies like AI and quantum computing \cite{shekariApplicationsTerahertzTechnology2025}. Developing more efficient, compact and cost-effective THz sources is vital. Current THz generation methods, including quantum cascade lasers and photoconductive antennas, have advanced significantly but still require further optimization to enhance their suitability for widespread and portable use \cite{THzSecurityApplications}. 

Photoconductive antennas in particular play an essential role in bridging the microwave-photonics-gap. Generating or detecting THz radiation with photoconductive antennas is advantageous because the devices work well at room temperature and under ambient conditions. Their versatility and tuning range is desirable in many applications. Photoconductors operate with laser powers in the milliwatts, enabling compact, fiber-coupled systems. These systems can be driven by equally compact and stable \num{1550} \si{\nano \meter} fiber lasers, making them well-suited for industrial applications \cite{naftalyIndustrialApplicationsTerahertz2019}. Many commercially available THz systems are based on broadband photoconductive antenna technology \cite{burfordReviewTerahertzPhotoconductive2017}. THz-TDS systems have the potential of making a significant impact in many research fields, such as X-ray technology, pharmaceutics, industrial monitoring etc. Industrial applications in particular demand high speed THz-TDS systems, capable of providing femtosecond resolution. A key variable in achieving high resolution are THz receivers and detectors that provide a high dynamic range (DNR). The development of photoconductive antennas that provide high DNR is thus majorly important. DNR can be influenced in many ways. Undoubtedly, material engineering and antenna design are two factors that can heavily influence a photoconductive antenna's performance.  

This thesis focuses on the improvement of the broadband THz performance of PCAs. PCAs used today often show unwanted resonant behavior in the time-domain, which limits the Fourier-transformation of the antennas and thus limits bandwidth. By combining two different materials for the antenna structure, namely gold and Nickel-Chromium, we hope to eliminate low frequency current spikes and time-harmonic resonances. Eliminating those subsequently improves the broadband behavior of a PCA. 