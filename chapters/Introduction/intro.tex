In between the microwave and infrared (IR) frequencies of the electromagnetic (EM) spectrum lies Terahertz (THz) radiation (make ref to figure) \cite{zhangIntroductionTHzWave2010}. THz radiation refers to the frequency (wavelength) spectrum ranging from \num{100} \si{\giga\hertz} (\num{3} \si{\milli\meter}) up to \num{10} \si{\tera\hertz} (\num{30} \si{\micro\meter}). While microwave and infrared sources are able to provide high magnitudes of power at those frequencies, there has been a lack of efficient and feasible high power sources in the THz range \cite{perkowitzNavigatingTerahertzGap2020}. This has led to the common reference to the THz range as the \enquote{THz gap} (e.g. see \cite{dhillon2017TerahertzScience2017, williamsFillingTHzGap2006, zhangAdvancesTerahertzTechnology2021}). THz radiation shows potential in many fields. Thus tremendous efforts have been made in the last two decades to narrow the \enquote{THz gap}. Narrowing the gap has been achieved from both the microwave and the optical side. Advances have been made by extending the high frequency cut-off in purely electronic RF devices, decreasing the lower cut-off frequencies of purely optical IR devices or even by combining the two approaches \cite{preuTunableContinuouswaveTerahertz2011}. Progress has been made not only driven by advances in THz technology but also a wide range of applications. One of the most important techniques using THz radiation is THz Time Domain Spectroscopy (THz-TDS). In many fields such as pharmaceutics \cite{huangProgressApplicationTerahertz2023}, materials sciences \cite{zhangApplicationTHzTDSCharacterization2024}, chemistry \cite{fischerChemicalRecognitionTerahertz2005} and many more (e.g. see \cite{petrovMobileNearfieldTerahertz2023, markelzPerspectiveTerahertzApplications2022,TerahertzSpectroscopyIts2011,kleine-ostmannReviewTerahertzCommunications2011}), THz-TDS has proven to be a valuable technique.

The THz frequency range exhibits several significant spectral features, including rotational transitions of gas-phase molecules, large-amplitude vibrational modes of organic compounds, lattice vibrations in solids, energy gaps in superconductors and intraband transitions in semiconductors \cite{PrinciplesTerahertzScience2009}. Techniques like THz-TDS exploit these characteristic material responses to probe and analyze the interaction of THz radiation with matter. Compared to the neighboring radio and infrared regions, the THz band exhibits much higher atmospheric opacity due to molecular rotational absorption lines \cite{fedorovPowerfulTerahertzWaves2020}. Water vapor plays a dominant role in attenuating THz radiation as it strongly absorbs energy in this frequency range. The unique spectral line structures of different molecular species allow for their identification within unknown samples. The shapes of these lines offer valuable insight into microscopic processes such as molecular collisions \cite{PrinciplesTerahertzScience2009}.

Since being introduced THz-TDS has been applied to many materials. Those include biomolecules, medicines, cancer tissue, DNA, proteins and bacteria \cite{chenLargeOxidationDependence2005,waltherNoncovalentIntermolecularForces2003,fischerTerahertzTimedomainSpectroscopy2005}. Here, THz-TDS can deliver valuable information IR spectroscopy cannot. One example is the ability to observe intermolecular vibrations in chemicals and organic molecules where the intramolecular mode appears in the IR region \cite{nagaiDirectEvidenceIntermolecular2005}. Intermolecular vibration studies using THz-TDS are expected to enhance our knowledge of larger biomolecules and the human body \cite{tonouchiCuttingedgeTerahertzTechnology2007}. 

THz-TDS can be combined with density functional theory \cite{chenCombinationTerahertzSpectroscopy2022} to study amino acids \cite{liaoAminoacidClassificationBased2023}, peptides \cite{neuTerahertzSpectroscopyTetrameric2019}, drugs \cite{kawaseNondestructiveTerahertzImaging2003} and explosives \cite{daviesTerahertzSpectroscopyExplosives2008}. THz radiation is transparent to most dry dielectric materials due to its relatively long wavelength. THz waves easily penetrate most clothing \cite{prokschaTerahertzInsightsFabric2024} or packaging \cite{wietzkeTerahertzSpectroscopyPolymers2011} material. This makes THz-TDS a valuable technique in security applications, quality and process controls and nondestructive analysis of materials and devices. THz radiations sensitivity to water can be used to control food and agricultural products \cite{afsah-hejriTerahertzSpectroscopyImaging2020}. For example, damage to fruits can be evaluated and the water content in vegetables can be monitored. Within the industrial food sector, compact and high-speed THz cameras capable of providing instant quality control information of products on conveyer belts are needed \cite{THzSecurityApplications}. \textbf{TODO: THz imaging / THz-TDS for cancer stuff, non ionizing etc.}

\textbf{Überleitung zu Photoconductors bzw. vernünftige THz-TDS signale um zur motivation überzuleiten}







