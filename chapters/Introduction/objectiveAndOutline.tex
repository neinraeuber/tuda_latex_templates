The objective of this thesis is to reduce low frequency resonances in H-Dipole PCAs when being used in pulsed operation THz systems. These resonances limit the bandwidth of a measured signals Fourier Transform and are thus a major problem in broadband apllications. Adding NiCr sections to our antenna feeds we hope to limit low frequency surface current resonances. Multiple configurations of the NiCr-induced feeds are first investigated in CST simulations before the antennas are fabricated ans investigated concerning their THz performace as photoconductive THz sources.  

Chapter 2 presents the theory needed for understanding THz-TDS and its challenges. Photoconductor theory, especially the principles of photomixing for pulsed operation and suitable antenna geometries for THz-TDS are presented. The cause for low frequency surface current resonances and their effects on THz-TDS are explained. Chapter 3 deals with designing and simulating an antenna that fulfills our needs. Suitable antenna topologies combined with NiCr-induced feeds are presented and simulated in CST Studio Suite. The simulation results are presented. Chapter 4 presents the fabrication process of the antennas to be investigated. The devices are also characterized in terms of their dark current (DC) IV-characteristics. Chapter 5 explains the TDS measurement setup, the steps taken for alignment and how the measurement data is to be analyzed. In chapter 6, the THz performance of the devices is investigated. We present the time domain and frequency domain signals and compare the effects of NiCr-induced feeds, as well as comparing the results to our simulations. Chapter 7 summarizes and concludes the work. An outlook on possible future work is given.  