The main objective of this thesis is to design, fabricate and experimentally evaluate photoconductive antennas capable of broadband operation and improved low-frequency characteristics. We aim to achieve the specific goal of suppressing undesired low-frequency resonances by incorporating resistive NiCr feeding segments into the PCAs. The work addresses the following research questions: 
\begin{itemize}
    \item Does the inclusion of NiCr segments in PCAs effectively suppress low-frequency surface current resonances?
    \item How do different NiCr configurations influence the antennas THz performance?
    \item Do the improvements predicted in simulations translate to actual measurements of fabricated devices in a THz-TDS setup?
\end{itemize}

In order to answer these questions, the thesis is structured as follows:


% A conventional photoconductive antenna, fabricated by depositing gold on a semiconducting substrate, is modified. Some part of the antenna is replaced by a high resistive Nickel-Chromium section. The effects of adding Nickel-Chromium are simulated in CST Studio Suite. Different configurations are simulated and compared in order to optimize performance. The most promising configurations are selected to be fabricated. After fabrication, the antennas are investigated in terms of their THz performance as photoconductive sources.

% By replacing some portion of an already existing photoconductive antenna geometry by Nickel-Chromium we hope to reduce low frequency resonances caused by the antenna pads. Additionally, we hope to reduce time-harmonic oscillations caused by the antenna feeding strip. Design optimization using CST Studio Suite means sweeping through a set of configurations to find optimal radiation characteristics. The fabrication of the antenna is the same as for a conventional photoconductive antenna with the additional step of depositing Nickel-Chromium. The THz performance is investigated using THz-TDS. The final goal of improving the low frequency characteristics of photoconductive antennas to improve bandwidth and signal to noise ratio has hopefully been accomplished. 

\textbf{Chapter 2} reviews the principles of THz-TDS and the operation of PCAs, especially the principles of photomixing for pulsed operation. Suitable antenna geometries are evaluated. The cause for low frequency surface current resonances and their effects on THz-TDS, as well as lower frequency time-harmonic resonances, are explained. Theoretical tools such as Fourier transformation and equivalent circuit models are introduced to describe these effects 
\textbf{Chapter 3} deals with designing and simulating an antenna that improves THz performance. A reference H-Dipole antenna is introduced. 
The reference H-Dipole is systematically modified with NiCr segments and simulated using CST Studio Suite. Surface current distributions and radiation impedance are analyzed in order to predict the influence of NiCr segments on the THz performance of PCAs. \textbf{Chapter 4} presents the lithography and metal deposition processes used to fabricate the reference devices, as well as NiCr-modified antennas. \textbf{Chapter 5} describes the THz measurement setup. The fabricated devices are DC characterized, followed by an evaluation of their THz performance. Measurement results are compared to simulations and the effectiveness of introducing  NiCr segments to suppress low-frequency resonances is evaluated. In
\textbf{chapter 6}, the main findings of this thesis are summarized. An outlook on possible further improvements is given. 

By integrating electromagnetic simulations and THz measurements, this thesis aims to demonstrate that resistive NiCr feeds are a viable and effective method for mitigating low-frequency resonances in PCAs. The approach is expected to contribute to the development of more broadband and robust THz devices.