The objective of this thesis is to design a photoconductive antenna capable of broadband operation and improved low frequency characteristics. A conventional photoconductive antenna, fabricated by depositing gold on a semiconducting substrate, is modified. Some part of the antenna is replaced by a high resistive Nickel-Chromium section. The effects of adding Nickel-Chromium are simulated in CST Studio Suite. Different configurations are simulated and compared in order to optimize performance. The most promising configurations are selected to be fabricated. After fabrication, the antennas are investigated in terms of their THz performance as photoconductive sources.

By replacing some portion of an already existing photoconductive antenna geometry by Nickel-Chromium we hope to reduce low frequency resonances caused by the antenna pads. Additionally, we hope to reduce time-harmonic oscillations caused by the antenna feeding strip. Design optimization using CST Studio Suite means sweeping through a set of configurations to find optimal radiation characteristics. The fabrication of the antenna is the same as for a conventional photoconductive antenna with the additional step of depositing Nickel-Chromium. The THz performance is investigated using THz-TDS. The final goal of improving the low frequency characteristics of photoconductive antennas to improve bandwidth and signal to noise ratio has hopefully been accomplished. 

Chapter 2 presents the theory needed for understanding THz-TDS and its challenges. Photoconductor theory, especially the principles of photomixing for pulsed operation and suitable antenna geometries for THz-TDS are presented. The cause for low frequency surface current resonances and their effects on THz-TDS, as well as lower frequency time-harmonic resonances, are explained. Chapter 3 deals with designing and simulating an antenna that improves THz performance. Suitable antenna topologies combined with Nickel-Chromium-induced feeds are presented and simulated in CST Studio Suite. The simulation results are presented. Chapter 4 presents the fabrication process of the antennas to be investigated. The devices are also characterized in terms of their dark current (DC) IV-characteristics. Chapter 5 explains the TDS measurement setup, the steps taken for alignment and how the measurement data analyzed. In chapter 6, the THz performance of the devices is investigated. We present the time domain and frequency domain signals and compare the effects of NiCr-induced feeds, as well as comparing the results to our simulations. Chapter 7 summarizes and concludes the work. An outlook on possible future work is given.  