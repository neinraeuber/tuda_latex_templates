%% This is file `DEMO-TUDaLetter.tex' version 4.01 (2025-02-11),
%% it is part of
%% TUDa-CI -- Corporate Design for TU Darmstadt
%% ----------------------------------------------------------------------------
%%
%% Copyright (C) 2018--2025 by Marei Peischl <marei@peitex.de>
%%
%% ============================================================================
%% This work may be distributed and/or modified under the
%% conditions of the LaTeX Project Public License, either version 1.3c
%% of this license or (at your option) any later version.
%% The latest version of this license is in
%% http://www.latex-project.org/lppl.txt
%% and version 1.3c or later is part of all distributions of LaTeX
%% version 2008/05/04 or later.
%%
%% This work has the LPPL maintenance status `maintained'.
%%
%% The current maintainer of this work is
%%   Marei Peischl <tuda-ci@peitex.de>
%%
%% The development repository can be found at
%% https://github.com/tudace/tuda_latex_templates
%% Please use the issue tracker for feedback!
%%
%% If you need a compiled version of this document, have a look at
%% http://mirror.ctan.org/macros/latex/contrib/tuda-ci/doc
%% or at the documentation directory of this package (if installed)
%% <path to your LaTeX distribution>/doc/latex/tuda-ci
%% ============================================================================
%%
% !TeX program = lualatex
%%

% Enable PDF/A via pdfmanagement and no longer via pdfx
\DocumentMetadata{
	pdfstandard=a-2b,
	pdfversion=1.7,% 2.0 is possible as well, but PDF/A-2b requires < 2.0
	lang=en,
}

\documentclass[
	english,% Main language as global option
	accentcolor=9c,% Choose accent color: For a list of available colors see the full tudapub documentation
%	logo=false,% Disable logo for all pages but first
	premium=true,%  enable coloring
%	firstpagenumber=false,% disable pagenumbering on first page
%	textwidth=narrow,% disable change of text width after the first page
%	accept-missing-logoes=true,% No error in case logo files are not available
%	logofile=example-image,% In case logo should be replaced
]{tudaletter}

%%%%%%%%%%%%%%%%%%%
% Language setup
%%%%%%%%%%%%%%%%%%%
\usepackage[english]{babel}
\usepackage[autostyle]{csquotes}% \enquote, to simplify use of quotation marks


\LoadLetterOption{DEMO-TUDaFromaddress}% Load address data from lco-file

\begin{document}


\begin{letter}{%
	Technische Universität Darmstadt\\%
	Referat Kommunikation\\%
	Karolinenplatz 5\\%
	64289 Darmstadt}

\setkomavar{subject}{\LaTeX{} letters using TU Darmstadt CI}
% \setkomavar{date}{2024-02-26}% In case \today should not be used
\setkomavar{yourmail}{yyyy-mm-dd}
\setkomavar{myref}{Demo-xx}
\setkomavar{frombank}{IBAN: 1234 5678 9123 4567 89}

\opening{Hello,}
this is a template file to show the usage of the tudaletter class.

The most important options can be found within the source file of this document.
Other options or layout adjustments may not comply CI regulations and therefore  should not be used.

Supported options:\\
\parbox{\linewidth}{
	\begin{description}
		\item[premium=true/false] Enable coloring of the headline. Disabled by default.
		\item[firstpagenumber=true/false] Disable pagenumbering for the first page. Enabled by default.
		\item[raggedright=true/false] Set all text ragged right. Default setting is false and uses justified text.
		\item[logo=true/false] Disable logo in the headline for all pages but first one.
		\item[texwidth=narrow/wide] The design requires the text width to change between the first page and the rest of the letter.
		\item[texwidth=narrow/wide] This setting is corresponds to the value “wide”.
		      I case this change happens within a paragraph the mechanism to do this adjustment is quite complex and might not work in all cases.
		      If you face any other special case it would be great to receive a report, so we can try to improve the used methods.
	\end{description}
}

To reuse the sender's address the template provides an example lco-file.
See the \KOMAScript{} documentation on details about these files and how to use them.

\closing{Happy \TeX{}ing}

\encl{Sources for this document:\\
	\texttt{DEMO-TUDaLetter.tex} (Letter),\\
	\texttt{DEMO-TUDaFromaddress.lco} (Address data)}

\end{letter}


\end{document}
%% End of file `DEMO-TUDaLetter.tex'.
