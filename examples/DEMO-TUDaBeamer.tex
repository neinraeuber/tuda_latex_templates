%% This is file `DEMO-TUDaBeamer2023.tex' version 3.37 (2024-04-21),
%% it is part of
%% TUDa-CI -- Corporate Design for TU Darmstadt
%% ----------------------------------------------------------------------------
%%
%%  Copyright (C) 2018--2024 by Marei Peischl <marei@peitex.de>
%%
%% ============================================================================
%% This work may be distributed and/or modified under the
%% conditions of the LaTeX Project Public License, either version 1.3c
%% of this license or (at your option) any later version.
%% The latest version of this license is in
%% http://www.latex-project.org/lppl.txt
%% and version 1.3c or later is part of all distributions of LaTeX
%% version 2008/05/04 or later.
%%
%% This work has the LPPL maintenance status `maintained'.
%%
%% The Current Maintainers of this work are
%%   Marei Peischl <tuda-ci@peitex.de>
%%
%% The development respository can be found at
%% https://github.com/tudace/tuda_latex_templates
%% Please use the issue tracker for feedback!
%%
%% If you need a compiled version of this document, have a look at
%% http://mirror.ctan.org/macros/latex/contrib/tuda-ci/doc
%% or at the documentation directory of this package (if installed)
%% <path to your LaTeX distribution>/doc/latex/tuda-ci
%% ============================================================================
%%
% !TeX program = lualatex
%%

\documentclass[
	english, % main language as globale option
%	logofile=example-image, % in case you are missing the logo files
	design=2023, % new slide design - will be default after official release
	]{tudabeamer}
	
\usepackage[main=english]{babel}

% Der folgende Block ist nur bei pdfTeX auf Versionen vor April 2018 notwendig
\usepackage{iftex}
\ifPDFTeX
\usepackage[utf8]{inputenc}%kompatibilität mit TeX Versionen vor April 2018
\fi


% document specific setup, you will not need that one in your files!
\newcommand*{\code}[1]{\texttt{#1}}

\title[TUDaBeamer2023]{\LaTeX~beamer using TU Darmstadts Corporate Design}
\subtitle{Design2023-beta}
\author[M. Peischl]{Marei Peischl}
\department{\TeX/\LaTeX}
\institute{pei\TeX}

% Scaling for additional logos and titlegraphic is described on slide “Scaling logos/titlegraphic”

% sublogo/partnerlogo will be placed in the headline/sidebar

\sublogo{\color{blue!50}\rule{\width}{\height}}
\partnerlogo*{\includegraphics{example-image}}

%  titlegraphic will be placed behind the title
%\titlegraphic*{\includegraphics{example-image}}
\date{2023-10-02}

\begin{document}

\maketitle

\tableofcontents

\section{Documentation}
\sectionpage

\begin{frame}{The tudabeamer class}
\begin{itemize}
	\item Basic usage identical to beamer
	\item No special syntax required
\end{itemize}
\end{frame}

\begin{frame}{Additional Features for titlegraphic/logos}
\begin{itemize}
	\item \code{\textbackslash{}partnerlogo} additional logo next to TUDa logo
	\item \code{\textbackslash{}sublogo} additional logo below TUDa logo
	\item \code{\textbackslash{}titlegraphic} place image as a background for title slide
\end{itemize}
	The additional option \code{authorontitle=true/false} allows to place author/institue below the title.
\end{frame}

\begin{frame}{Scaling logos/titlegraphic}
\begin{itemize}
\item The commands shown on previous slide support \code{\textbackslash{}height}/\code{\textbackslash{}width} to adjust the size to be used for graphic scaling.
\item All have a starred variant (e.\,g. \code{\textbackslash{}sublogo*\{\textbackslash{}includegraphics\{example-image\}\}}). This will automaticcaly scale/clip the logo/image to desired size.

To align the sublogo together with the visible part of the TUDa logo there also exists the possibility to add a second star behind the mandatory argument. This will move the logo to the left to match. This variant is only available for the sublogo.
\end{itemize}
\end{frame}

\begin{frame}{Table of Contents}
By design the toc is structured in two columns. This layout doesn't really support subsections or subsubsections.

To be able to use the new design with old content, the mechanism was extended to support those. This will automatically switch to a single columns toc.

In case you want to disable that mechanism there is the class option \code{toc-columns=one/two/auto} which can also be adjusted locally be directly setting it with the command: \code{\textbackslash{}tableofcontents[columns=two]}.

Please be also aware of beamer's mechanism to filter the toc, e.g. by \code{\textbackslash{}tableofcontents[hideallsubsections]}, see the beamer documentation for details.
\end{frame}

\section{Slide examples}

\begin{frame}{Frame with a subtitle}
\framesubtitle{Subtitle}
An example.
\end{frame}

\begin{frame}[fragile,uppercase=false]{Please be aware of uppercase frametitles}
The mechanism to provide uppercase frame titles might have strange side effects.
It's aware of macros, so those usually should work, but there are issues with arguments. e.g. using \verb+\color{dark2}+ will throw an error message.

You can avoic this by either locally disabling the case change:

\begin{verbatim}
\frameztitle{Test \NoCaseChange{\textcolor{dark2}{\MakeUppercase{Test}}}}
\end{verbatim}
or use the frame option \verb+uppercase=false+ as for this frame or the class option \verb+uppercase-frametitle=false+
\end{frame}

\textcolor{TUDa-dark1}{test}

\begin{frame}{Color Scheme}
The new design introduces new colors to be used to highlight content.

The additional colors are:
\par
\UseName{clist_map_inline:nn} {
	dark1,
	dark2,
	light1,
	light2,
	accent1,
	accent2,
	accent3,
	accent4,
	accent5,
	accent5,
	hyperlink,
} {TUDa-####1: \textcolor{TUDa-####1}{\rule{1cm}{\ht\strutbox}}\par}
\end{frame}

\begin{frame}{Remark on vertical alignment (especially columns)}
	The default alignment has been changed from \code{c} to \code{t} in contrast to beamers default setting.
	This means frame content will be aligned top inside the frame. 
	This is required by the design but might have side effects with the columns environment:
	
	To resolve this one can either globally add the \code{c} option or pass the option to the columns environment, e.\,g.:

\begin{columns}[onlytextwidth,c]
	\column{.8\linewidth}
	\begin{itemize}
		\item one
		\item two
	\end{itemize}
	\column{.2\linewidth}
	\includegraphics[width=\linewidth]{example-image}
\end{columns}
\end{frame}

\end{document}

