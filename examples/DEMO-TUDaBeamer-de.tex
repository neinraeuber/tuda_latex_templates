%% This is file `DEMO-TUDaBeamer-de.tex' version 4.03-dev (2025-03-13),
%% it is part of
%% TUDa-CI -- Corporate Design for TU Darmstadt
%% ----------------------------------------------------------------------------
%%
%% Copyright (C) 2018--2025 by Marei Peischl <marei@peitex.de>
%%
%% ============================================================================
%% This work may be distributed and/or modified under the
%% conditions of the LaTeX Project Public License, either version 1.3c
%% of this license or (at your option) any later version.
%% The latest version of this license is in
%% http://www.latex-project.org/lppl.txt
%% and version 1.3c or later is part of all distributions of LaTeX
%% version 2008/05/04 or later.
%%
%% This work has the LPPL maintenance status `maintained'.
%%
%% The current maintainer of this work is
%%   Marei Peischl <tuda-ci@peitex.de>
%%
%% The development repository can be found at
%% https://github.com/tudace/tuda_latex_templates
%% Please use the issue tracker for feedback!
%%
%% If you need a compiled version of this document, have a look at
%% http://mirror.ctan.org/macros/latex/contrib/tuda-ci/doc
%% or at the documentation directory of this package (if installed)
%% <path to your LaTeX distribution>/doc/latex/tuda-ci
%% ============================================================================
%%
% !TeX program = lualatex
%%
%% Falls PDF/A gebraucht wird:
% \DocumentMetadata{
% pdfstandard=a-2b,
% pdfversion=1.7,% 2.0 geht auch, aber dann geht kein PDF/A-2b
% lang=de,
%}

\documentclass[
	german,% Hauptsprache als globale Option, früher war ngerman notwendig
	accentcolor=9c,% Auswahl der Akzentfarbe
%	accept-missing-logos=true,% Keine Fehlermeldung, falls die Logo Dateien nicht vorliegen
%	logofile=example-image,% Falls die Logo-Datei ersetzt werden soll
]{tudabeamer}

%%%%%%%%%%%%%%%%%%%
% Spracheinstellungen
%%%%%%%%%%%%%%%%%%%
\usepackage[german]{babel}
\usepackage[autostyle]{csquotes}% Sprachabhängige Anführungszeichen mit \enquote

%%%%%%%%%%%%%%%%%%%
% Formatierungen für Beispiele in diesem Dokument. Im Allgemeinen nicht notwendig!
%%%%%%%%%%%%%%%%%%%
\newcommand*{\code}[1]{\texttt{#1}}
%%%%%%%%%%%%%%%%%%%
% Ende der Demo-Formatierungseinstellungen
%%%%%%%%%%%%%%%%%%%

\title[TUDaBeamer2023]{\LaTeX~beamer im Corporate Design der TU Darmstadt}
\subtitle{Design2023-beta}
\author[M. Peischl]{Marei Peischl}
\department{\TeX/\LaTeX}
\institute{pei\TeX}

% Die Skalierung von zusätzlichen Logos und Titelgrafiken ist auf der Folie „Skalierung von Logos/Titelgrafiken“ beschrieben
% In der Überschrift/Seitenleiste wird das Sublogo/Partnerlogo platziert

\sublogo{\color{blue!50}\rule{\width}{\height}}
\partnerlogo*{\includegraphics{example-image}}

% Die Titelgrafik wird hinter dem Titel platziert
% \titlegraphic*{\includegraphics{example-image}}

\date{2023-10-02}% Falls kein Datum angegeben ist, wird \today verwendet

\AtBeginSection{\sectionpage}% Abschnittstrenner aktivieren

\begin{document}

\maketitle

\tableofcontents

\section{Dokumentation}

\begin{frame}{Die Dokumentenklasse beamer}
	\begin{itemize}
		\item Grundlegende Verwendung ist wie bei beamer
		\item Keine spezielle Syntax erforderlich
	\end{itemize}
\end{frame}

\begin{frame}{Zusätzliche Funktionen für Titelgrafiken/Logos}
	\begin{itemize}
		\item \code{\textbackslash{}partnerlogo} zusätzliches Logo neben dem TUDa-Logo
		\item \code{\textbackslash{}sublogo} zusätzliches Logo unter dem TUDa-Logo
		\item \code{\textbackslash{}titlegraphic} platziert das Bild als Hintergrund der Titelfolie
	\end{itemize}
	Die zusätzliche Option \code{authorontitle=true/false} ermöglicht es, den Autor/das Institut unterhalb des Titels zu platzieren.
\end{frame}

\begin{frame}{Skalierung von Logos/Titelgrafiken}
	\begin{itemize}
		\item Die auf der vorherigen Folie gezeigten Befehle helfen \code{\textbackslash{}height}/\code{\textbackslash{}width} bei der Anpassung der Größe, die für die Grafikskalierung verwendet werden soll.
		\item Alle haben eine Variante mit Sternchen (z.\,B. \code{\textbackslash{}sublogo*\{\textbackslash{}includegraphics\{example-image\}\}}). Dadurch wird das Logo bzw. die Grafik automatisch auf die gewünschte Größe skaliert.

		      Damit das Sublogo an den sichtbaren Teil des TUDa-Logos angepasst werden kann, kann hinter dem Pflichtargument ein zweiter Stern hinzugefügt werden. Dies verschiebt das Logo entsprechend nach links. Diese Variante existiert nur für das Sublogo.
	\end{itemize}
\end{frame}

\begin{frame}{Inhaltsverzeichnis}
	Das Inhaltsverzeichnis ist in zwei Spalten unterteilt. Dieses Layout unterstützt keine subsections oder subsubsections.

	Um das neue Design mit alten Inhalten nutzen zu können wurde der Mechanismus erweitert. Dieser schaltet automatisch auf ein einspaltiges Inhaltsverzeichnis um.

	Falls dieser Mechanismus deaktiviert werden soll, dann gibt es die Klassenoption \code{toc-columns=one/two/auto}. Diese kann auch lokal angepasst werden mit dem Befehl:
	\code{\textbackslash{}tableofcontents[columns=two]}.

	Hierzu kann auch der beamer Mechanismus zum Filtern des Inhaltsverzeichnisses, z.\,B. \code{\textbackslash{}tableofcontents[hideallsubsections]} genutzt werden. Weitere Informationen dazu sind in der beamer Dokumentation.
\end{frame}

\section{Beispiele für Folien}

\begin{frame}{Folie mit Untertitel}
	\framesubtitle{Untertitel}
	Ein Beispiel.
\end{frame}

\begin{frame}[fragile,uppercase=false]{Die Titel der Folien werden in Großbuchstaben gesetzt}
	Der Mechanismus kann unerwartete Nebenwirkungen haben.
	Makros sollten normalerweise funktionieren, aber es gibt Probleme mit Argumenten, zum Beispiel bei der Verwendung von \verb+\color{dark2}+ wird eine Fehlermeldung erzeugt, weil die Farbe in Kleinbuchstaben definiert wurde.

	Dies kann vermieden werden, indem entweder die Groß-/Kleinschreibung lokal deaktiviert wird
\begin{verbatim}
\frameztitle{Test \NoCaseChange{\textcolor{dark2}{\MakeUppercase{Test}}}}
\end{verbatim}
	oder aber die Folienoption \verb+uppercase=false+ wie er für diese Folie  verwendet wird.
	Es ist auch möglich die Einstellung global zu ändern, hierfür wurde die Klassenoption \verb+uppercase-frametitle=false+ definiert.
\end{frame}

\begin{frame}{Farbschema}
	Das neue Design wurde um einige neue Farben ergänzt. Diese können zur Hervorhebung von Inhalten verwendet werden.

	Die zusätzlichen Farben sind:

	\renewcommand*{\do}[1]{TUDa-####1: \textcolor{TUDa-####1}{\rule{1cm}{\ht\strutbox}}\par}
	\docsvlist{dark1,dark2,light1,light2,accent1,accent2,accent3,accent4,accent5,accent5,hyperlink}
\end{frame}

\begin{frame}{Hinweise zur vertikalen Ausrichtung (insbesondere mit columns)}
	Die Standardausrichtung wurde im Gegensatz zur Standardeinstellung von beamer von \code{c} auf \code{t} geändert.
	Das neue Design verlangt, dass der Inhalt auf den Folien von oben nach unten platziert wird.
	Allerdings hat diese Änderung den Nebeneffekt, dass sich die vertikale Ausrichtung aller Objekte verändert.

	Um auftrendende Probleme zu vermeiden, kann entweder die Option \code{c} für einen frame global gesetzt werden, oder bei der Spaltenumgebung direkt.

	\begin{columns}[onlytextwidth,c]
		\column{.8\linewidth}
		\begin{itemize}
			\item Eins
			\item Zwei
		\end{itemize}
		\column{.2\linewidth}
		\includegraphics[width=\linewidth]{example-image}
	\end{columns}
\end{frame}

\end{document}
%% End of file `DEMO-TUDaBeamer-de.tex'.
