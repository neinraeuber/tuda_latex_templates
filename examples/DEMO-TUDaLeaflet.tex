%% This is file `DEMO-TUDaLeaflet.tex' version 4.00 (2025-01-26),
%% it is part of
%% TUDa-CI -- Corporate Design for TU Darmstadt
%% ----------------------------------------------------------------------------
%%
%% Copyright (C) 2018--2025 by Marei Peischl <marei@peitex.de>
%%
%% ============================================================================
%% This work may be distributed and/or modified under the
%% conditions of the LaTeX Project Public License, either version 1.3c
%% of this license or (at your option) any later version.
%% The latest version of this license is in
%% http://www.latex-project.org/lppl.txt
%% and version 1.3c or later is part of all distributions of LaTeX
%% version 2008/05/04 or later.
%%
%% This work has the LPPL maintenance status `maintained'.
%%
%% The current maintainer of this work is
%%   Marei Peischl <tuda-ci@peitex.de>
%%
%% The development repository can be found at
%% https://github.com/tudace/tuda_latex_templates
%% Please use the issue tracker for feedback!
%%
%% If you need a compiled version of this document, have a look at
%% http://mirror.ctan.org/macros/latex/contrib/tuda-ci/doc
%% or at the documentation directory of this package (if installed)
%% <path to your LaTeX distribution>/doc/latex/tuda-ci
%% ============================================================================
%%
% !TeX program = lualatex
%%

\documentclass[
	english,% Main language as global option
	accentcolor=9c,% Choose accent color: For a list of available colors see the full tudapub documentation
%	accept-missing-logoes=true,% No error in case logo files are not available
%	logofile=example-image,% In case logo should be replaced
]{tudaleaflet}

%%%%%%%%%%%%%%%%%%%
% Language setup
%%%%%%%%%%%%%%%%%%%
\usepackage[english]{babel}
\usepackage[autostyle]{csquotes}% \enquote, to simplify use of quotation marks

%%%%%%%%%%%%%%%%%%%
% Bibliography
%%%%%%%%%%%%%%%%%%%
\usepackage{biblatex}
\addbibresource{DEMO-TUDaBibliography.bib}% File name of BibTeX database

%%%%%%%%%%%%%%%%%%%
% Document specific setup for demonstration, generally not needed!
%%%%%%%%%%%%%%%%%%%
\newcommand*{\file}[1]{\texttt{#1}}
\newcommand*{\code}[1]{\texttt{#1}}
\newcommand*{\pkg}[1]{\textsf{#1}}
\newcommand*{\cls}[1]{\textsf{#1}}
\let\tbs\textbackslash
%%%%%%%%%%%%%%%%%%%
% End of demo specific setup
%%%%%%%%%%%%%%%%%%%

\begin{document}

\title{TUDaLeaflet}
% \titleimage{\includegraphics[width=\width,height=\height]{example-image}}
\subtitle{Leaflet using TUDa-CI}
\addTitleBox{Institute}

\AddSponsor{\includegraphics[height=\height ]{example-image}}
\AddSponsor{\includegraphics[height=\height ]{example-image}}
\AddSponsor{\includegraphics[height=\height ]{example-image}}

\maketitle

\section{Basic usage}
\cls{tudaleaflet} is based on the \cls{leaflet} class.
The usage is similar to the article class, only the page breaks are those of a leaflet and are normally divided into six partial pages.
Options not defined by \cls{tudaleaflet} may be passed to the base class.
See \cite{leaflet} for more information about the \cls{leaflet} class.

\pkg{scrextend} has added some mechanisms for better compatibility. However, this is not a \KOMAScript class, which is why not all mechanisms are available \cite[see][]{scrguide}.

\section{title page}
The title page is like the other TUDa CI classes and contains the following macros:
\begin{description}
	\item[title] Title
	\item[subtitle] Is placed in the block between the title and cover image
	\item[author] Can be placed under the subtitle if necessary
	\item[date] Can be placed under the author
	\item[addTitleBox] Title box
	      The title box is placed below the logo at a specified distance. The background is white and can be filled with text or institute logo.

	      With \code{\addTitleBoxLogo} logos can be used in the title box.
	      It accepts an image file path as an argument.
\begin{verbatim}
\addTitleBoxLogo{example-image}
\addTitleBoxLogo*{
\includegraphics
[width=\linewidth]
{example-image}
}
\end{verbatim}
	\item[titleimage] \code{titleimage} is usually for placing a graphic, but also accepts any content. The content is placed flush with the top corner in the body of the front page.
\begin{verbatim}
\titleimage{
\includegraphics
[width=\width]
{example-image}
}
\end{verbatim}
	      To fill the area there, these locally defined macros \verb+\height+ and \verb+\width+ will help.
\end{description}

\subsection{Title page customization options}
\begin{description}
	\item[colorback] Activates backround color of the \verb+\titleimage+ area. Default is \verb+color+""\verb+back=true+
	\item[colorbacktitle] Similar to title block. Default is \verb+colorbacktitle=false+
	\item[colorbacksubtitle] This option can only be used if \verb+colorbacktitle=true+. If \verb+colorbacktitle=false+ an error message occurs. The strip between the title block and the image area will be colored the same color as the identity bar. The default is \verb+colorbacksubtitle=false+
	\item[logofile] Alternate main logo, e.\,g. in case the TUDa logo is not available.
\end{description}

Text alignment of the title is done with \verb+\raggedtitle+.
To prevent breaks within the title, the default is \verb+\raggedright+.
Other alignments can be done with \verb+\renewcommand+.

\subsection{Sponsor logos}
\begin{verbatim}
\AddSponsor{<logo1>}
\AddSponsor{<logo2>}
\end{verbatim}
is available for sponsor logos.
In the example, all logos are at the same height. This can be done with \verb+\height+ within the argument.
The block always ends with the text on the sides. The space between the logos is automatically filled.

The second variant can be particularly helpful for logos with different heights.
It enables manual vertical alignment, where only the spacing and separating lines are added to the logos:

\begin{verbatim}
\sponsors{
<logo1><logo2>
}
\end{verbatim}

Usually the logos are at the bottom of the title page. However, the CI guidelines also offer the option to place them at the end of the document.

TUDaLeaflet provides \verb+sponsor+ with the values \verb+title+, \verb+lastfoot+ or \verb+manual+.
The default is \verb+title+.
\verb+lastfoot+ places the sponsors on the last page. This is done with \verb+\vfill+.
\verb+manual+ in turn allows the box to be placed manually.

\begin{verbatim}
\insertSponsors
\end{verbatim}
Is the corresponding macro for this and inserts the box at the appropriate position.
But this only works if \verb+sponsor=manual+.
Otherwise it has no effect.


\printbibliography

\end{document}
%% End of file `DEMO-TUDaLeaflet.tex'.
