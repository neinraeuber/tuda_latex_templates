%% This is file `DEMO-TUDaPoster.tex' version 4.02 (2025-02-25),
%% it is part of
%% TUDa-CI -- Corporate Design for TU Darmstadt
%% ----------------------------------------------------------------------------
%%
%% Copyright (C) 2018--2025 by Marei Peischl <marei@peitex.de>
%%
%% ============================================================================
%% This work may be distributed and/or modified under the
%% conditions of the LaTeX Project Public License, either version 1.3c
%% of this license or (at your option) any later version.
%% The latest version of this license is in
%% http://www.latex-project.org/lppl.txt
%% and version 1.3c or later is part of all distributions of LaTeX
%% version 2008/05/04 or later.
%%
%% This work has the LPPL maintenance status `maintained'.
%%
%% The current maintainer of this work is
%%   Marei Peischl <tuda-ci@peitex.de>
%%
%% The development repository can be found at
%% https://github.com/tudace/tuda_latex_templates
%% Please use the issue tracker for feedback!
%%
%% If you need a compiled version of this document, have a look at
%% http://mirror.ctan.org/macros/latex/contrib/tuda-ci/doc
%% or at the documentation directory of this package (if installed)
%% <path to your LaTeX distribution>/doc/latex/tuda-ci
%% ============================================================================
%%
% !TeX program = lualatex
%%

\documentclass[
	english,% Main language as global option
	paper=a0,% The default paper format is a0
	accentcolor=9c,% Choose accent color: For a list of available colors see the full tudapub documentation
	logo=body,% Place logo below identity bar
	footer=true,
%	accept-missing-logoes=true,% No error in case logo files are not available
%	logofile=example-image,% In case logo should be replaced
]{tudaposter}

%%%%%%%%%%%%%%%%%%%
% Language setup
%%%%%%%%%%%%%%%%%%%
\usepackage[english]{babel}
\usepackage[autostyle]{csquotes}% \enquote, to simplify use of quotation marks

%%%%%%%%%%%%%%%%%%%
% Document specific setup for demonstration, generally not needed!
%%%%%%%%%%%%%%%%%%%
\newcommand*{\file}[1]{\texttt{#1}}
\newcommand*{\code}[1]{\texttt{#1}}
\let\tbs\textbackslash
\usepackage{multicol}
%%%%%%%%%%%%%%%%%%%
% End of demo specific setup
%%%%%%%%%%%%%%%%%%%

\begin{document}

\title{\LaTeX{} using TU Darmstadt's Corporate Design}
\subtitle{The document class tudaposter}
\author{Marei Peischl\thanks{pei\TeX{} \TeX{}nical Solutions}\and der \TeX-Löwe}

\titlegraphic{\color{red!20}\rule{\contentwidth}{.3\contentheight}}

\addTitleBox{titlebox}
\addTitleBox{titlebox 2}
\footerqrcode{https://peitex.de}
\footer{Footer content}% If activated

\maketitle

\section*{General information}

The document class tudaposter is part of the TUDa-CI bundle. It is for posters and non-scientific posters using TU Darmstadts Corporate Design.

\begin{multicols}{2}
	\subsection*{Usage}
	Basic usage is nearly identical to tudapub, as the titles are structured similarly.
	Like tudapub, the tudaposter class is based on \KOMAScript{}.
	This has the advantage that more can be used than would be necessary for a poster.

	tudaposter makes it possible to write continuous text.
	The tudasciposter class, which is based on tikzposter, cannot do this.

	\subsection*{Title}
	The title can be set with \code{\maketitle}.
	The macros \code{\title}, \code{\subtitle} and \code{\author} can be used here but also available are \code{\titlegraphic}, \code{\addTitleBox}, \code{\footerqrcode} and \code{\footer}. Except for \code{\footerqrcode} only the content is placed accordingly.
	An example of usage is shown in DEMO-TUDaPoster.tex.

	\subsection*{Specify lengths}
	Two length dimensions \texttt{\textbackslash{}contentwidth} and \texttt{\textbackslash{}contentheight} are predefined within the poster content (this also includes the title graphic).
	The height here is the distance between the title block and the footer/divider line.

	\subsection*{Document class options}
	\begin{description}
		\item[paper=<papersize>] Formats from A0 to A4 are supported. Default is a0.
		\item[fontsize=<fontsize/auto>] The base font size is defined here, so that the other sizes then scale accordingly. The default setting is \code{auto}. It is for scaling increments on supported paper sizes.
		\item[logo=head/body] The logo will appear in the header or within the document content.
		\item[color=<colorcode>] Accent color according to corporate design guidelines.
		\item[colorsubtitle=true/false] The background color for the subtitle can be switched on and off.
		\item[footer=true/false] Default is \code{false}.
		\item[marginpar=true/false] The width of marginpar is the width of the logo. This can be filled with \code{\tbs{}SetMarginpar} or \code{\tbs{}marginpar}. For more details see \file{DEMO-TUDaAnnouncement}.

		      The image that belongs to the title is overlapped with marginpar and therefore \code{marginpar=true} and \code{\tbs{}titlegraphic} should only be used carefully together.
		\item[title=large/small/default] Reduces the font size by one level. In this case \code{large} is \code{default} and \code{small} choose the title font size of the next smaller paper size where the base font size and marginpar stay the same.
		\item[type=default/announcement] Different layouts of posters can be loaded here.
		      \code{announcement} sets the options \code{marginpar=true,indenttext=false,logo=head,title=small,colorsubtitle=true} and also activates the output of the subtitle in bold font.
	\end{description}
\end{multicols}


\end{document}
%% End of file `DEMO-TUDaPoster.tex'.
