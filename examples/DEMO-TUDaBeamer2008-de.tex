%% This is file `DEMO-TUDaBeamer2008-de.tex' version 4.00 (2025-01-26)
%% it is part of
%% TUDa-CI -- Corporate Design for TU Darmstadt
%% ----------------------------------------------------------------------------
%%
%% Copyright (C) 2018--2025 by Marei Peischl <marei@peitex.de>
%%
%% ============================================================================
%% This work may be distributed and/or modified under the
%% conditions of the LaTeX Project Public License, either version 1.3c
%% of this license or (at your option) any later version.
%% The latest version of this license is in
%% http://www.latex-project.org/lppl.txt
%% and version 1.3c or later is part of all distributions of LaTeX
%% version 2008/05/04 or later.
%%
%% This work has the LPPL maintenance status `maintained'.
%%
%% The Current Maintainers of this work are
%%   Marei Peischl <tuda-ci@peitex.de>
%%
%% The development respository can be found at
%% https://github.com/tudace/tuda_latex_templates
%% Please use the issue tracker for feedback!
%%
%% If you need a compiled version of this document, have a look at
%% http://mirror.ctan.org/macros/latex/contrib/tuda-ci/doc
%% or at the documentation directory of this package (if installed)
%% <path to your LaTeX distribution>/doc/latex/tuda-ci
%% ============================================================================
%%
% !TeX program = lualatex
%%

\documentclass[
	german,% Globale Übergabe der Hauptsprache
	aspectratio=169,% Beamer Option zum Umschalten des Formates
	accentcolor=2d,% Akzentfarbe
	logo=false,% Kein Logo auf Folgeseiten
	colorframetitle=true,% Akzentfarbe auch im Frametitle
%	logofile=example-image, % Falls die Logo Dateien nicht vorliegen
	design=2008,  %Um das alte Design zu aktivieren
]{tudabeamer}
\usepackage[main=german]{babel}

% Der folgende Block ist nur bei pdfTeX auf Versionen vor April 2018 notwendig:
\usepackage{iftex}
\ifPDFTeX
	\usepackage[utf8]{inputenc}% Kompatibilität mit TeX Versionen vor April 2018
\fi


% Makros für Formatierungen der Doku
% Im Allgemeinen nicht notwendig!
\let\code\texttt

\title{LaTeX-Beamer im Corporate Design der TU Darmstadt}
\subtitle{Version 1.2}
\author[M. Peischl]{Marei Peischl}
\department{\TeX/\LaTeX}
\institute{pei\TeX}

% Fremdlogo
% Logo MaKro mit Sternchen skaliert automatisch, sodass das Logo in die Fußzeile passt
\logo*{\includegraphics{example-image-16x9}}

% Da das Bild frei wählbar nach Breite und/oder Höhe skaliert werden kann, werden \width/\height entsprechend gesetzt. So kann die Fläche optimal gefüllt werden.
% Sternchenversion skaliert automatisch und beschneidet das Bild, um die Fläche zu füllen.
\titlegraphic*{\includegraphics{example-image}}
\date{15. Mai 2019}

\begin{document}

\maketitle

\section{Dokumentation}
\begin{frame}{Die Dokumentenklasse tudabeamer}
	\begin{itemize}
		\item Verwendung wie beamer
		\item Keine besondere Syntax notwendig
		\item Klassenoption accentcolor wählt Akzentfarbe
		\item Option serif=true für Serifen
	\end{itemize}
\end{frame}

\begin{frame}{Zusätzliche Features der Titelfolie}
	\begin{itemize}
		\item \code{\textbackslash{}logo} wählt Fremdlogo für Fußzeile
		\item \code{\textbackslash{}titlegraphic} ersetzt den unteren Teil der Titelfolie. Zusätzlich existiert \code{\textbackslash{}titlegraphic*{Inhalt}}.
		      In diesem Fall wird der Inhalt in eine Box gesetzt, die so skaliert wird, dass sie den Bereich des Titelbildes überdeckt und entsprechend mittig ausgeschnitten ist.
	\end{itemize}
	Über die Option \code{authorontitle=true/false} kann zu den offiziellen Vorgaben auch der Autor und das Institut groß auf der Titelfolie gezeigt werden.
\end{frame}

\setupTUDaFrame{logo=true}
\begin{frame}[fragile]{Logo im Frametitle}
	Das Logo innerhalb des Frametitle kann mit der Klassenoption \code{logo=false} abgeschaltet werden.

	Soll das Logo später für ein Folie oder einen Bereich wieder aktiviert werden, steht das Makro
\begin{verbatim}
\setupTUDaFrame{logo=true}
\end{verbatim}
	zur Verfügung. Dort kann die globale Einstellung lokal überschrieben werden.
\end{frame}

\begin{frame}{Frame mit Untertitel}
	\framesubtitle{Untertitel}
	Ein Beispiel.
\end{frame}

\begin{frame}{Blöcke}
	\begin{block}{Standardblock mit Titel}
		Blockinhalt
	\end{block}
	\begin{block}{}
		Ohne Titel
	\end{block}
\end{frame}

\begin{frame}{Spezielle Blöcke}
	\begin{exampleblock}{Exampleblock}
		Blockinhalt
	\end{exampleblock}
	\begin{alertblock}{Alertblock}
		Blockinhalt
	\end{alertblock}
	\begin{example}[Für die example-Umgebung]
		Inhalt
	\end{example}
\end{frame}

\begin{frame}{Hinweis zur Ausrichtung (insbesondere columns)}
	Die Standardausrichtung wurde gegenüber den Beamer-Voreinstellungen von \code{c} zu \code{t} geändert. Dies bedeutet, dass Inhalt auf der Folie oben ausgerichtet wird. Dies entspricht den Vorgaben. Es hat allerdings den Nachteil, dass die \code{columns}-Umgebung in diesem Fall bei der Positionierung von Bildern ungewohnte Ergebnisse erzeugt.

	Die Ausrichtung kann in diesem Fall entweder global mit der Option \code{c} wieder zum Standard geändert werden oder aber das \code{c} wird direkt an die \code{columns}-Umgebung übergeben. Zum Beispiel:
	\begin{columns}[onlytextwidth,c]% Ohne das c ist die Ausrichtung verschoben
		\column{.8\linewidth}
		\begin{itemize}
			\item eins
			\item zwei
		\end{itemize}
		\column{.2\linewidth}
		\includegraphics[width=\linewidth]{example-image}
	\end{columns}
\end{frame}


\begin{frame}[fragile]{Anpassungen der Mathematikschriftarten}
	Es gibt keine feste Vorgabe zur Verwendung einer Mathematikschrift.

	In der Diskussion (\url{https://github.com/tudace/tuda_latex_templates/issues/30}) hat sich folgendes als hinreiche Lösung herausgestellt. Jedoch funktioniert diese Lösung nicht in pdflatex!
\begin{verbatim}
	\usepackage{unicode-math}
	\setmathfont{Fira Math}
	\setmathfont[range=up]{Roboto}
	\setmathfont[range=it]{Roboto-Italic}
	\setmathfont[range=\int]{Fira Math}
\end{verbatim}
	Allgemein kann die Mathematikschriftart natürlich auch durch Pakete angepasst werden.
\end{frame}


\begin{frame}[fragile]{Spezielle Anpassungen des Fachbereichs Maschinenbau}
	Mit Version 3.0 wird TUDa-CI um die Anpassungen des Corporate Designs des Fachbereichs Maschinenbau ergänzt.
	Daher verfügt tudabeamer nun ebenfalls über die Option
\begin{verbatim}
	department=mecheng
\end{verbatim}
	oder kurz
\begin{verbatim}
	mecheng
\end{verbatim}
	Dieser Modus setzt automatisch alle notwendigen Änderungen, benötigt jedoch die zusätzlichen Logos.

	Falls das Logo des Fachbereichs nicht vorliegt, kann durch die Option
\begin{verbatim}
	departmentlogo=example-image
\end{verbatim} ein Beispielbild statt des Logos verwendet werden.
\end{frame}

\begin{frame}[fragile]{mecheng: Hintergrundfarbe}
	Diese Variante des Corporate Designs erlaubt es die Hintergrundfarbe der einzelnen Folien zu verändern. Hierfür wurden die Frames um die Option \code{bgcolor} erweitert. Um in die beiden voreingestellten Modi zu wechseln, gibt es die Konfiguration
\begin{verbatim}
	bgcolor=Primary1
\end{verbatim}
	oder
\begin{verbatim}
	bgcolor=Primary2
\end{verbatim}
	Damit wird auch die Textfarbe entsprechend der Vorgaben gewählt. Bei Auswahl einer abweichenden Farbe bleibt die Textfarbe unverändert und muss ggf. manuell angepasst werden.
\end{frame}

\begin{frame}[fragile]{Änderung an der Option color in Version 3.33.}
	Bis version 3.33 konnte tudabeamer mit der Option \verb+color={Optionen}+ Optionen an das Paket tudacolors übergeben.

	Aufgrund eines Konflikts mit dem siunitx Paketes, welches jedoch weiter verbreitet ist als TUDa-CI wurde für Version 3.33 die Option so umbenannt, dass nun der Optionsschlüssel \verb+tudacolors={Optionen}+ notwendig ist.
	Siehe auch \url{https://github.com/tudace/tuda_latex_templates/issues/435}

	Die Funktion funktioniert weiterhin wie gewohnt, allerdings wird eine Warnung erzeugt. Falls siunitx geladen und das Problem tatsächlich auftritt, wird diese Warnung vor der erzeugten Fehlermeldung ausgegeben.
\end{frame}

\end{document}
