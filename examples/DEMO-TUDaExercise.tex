%% This is file `DEMO-TUDaExercise.tex' version 4.01-dev (2025-02-11),
%% it is part of
%% TUDa-CI -- Corporate Design for TU Darmstadt
%% ----------------------------------------------------------------------------
%%
%% Copyright (C) 2018--2025 by Marei Peischl <marei@peitex.de>
%%
%% ============================================================================
%% This work may be distributed and/or modified under the
%% conditions of the LaTeX Project Public License, either version 1.3c
%% of this license or (at your option) any later version.
%% The latest version of this license is in
%% http://www.latex-project.org/lppl.txt
%% and version 1.3c or later is part of all distributions of LaTeX
%% version 2008/05/04 or later.
%%
%% This work has the LPPL maintenance status `maintained'.
%%
%% The current maintainer of this work is
%%   Marei Peischl <tuda-ci@peitex.de>
%%
%% The development repository can be found at
%% https://github.com/tudace/tuda_latex_templates
%% Please use the issue tracker for feedback!
%%
%% If you need a compiled version of this document, have a look at
%% http://mirror.ctan.org/macros/latex/contrib/tuda-ci/doc
%% or at the documentation directory of this package (if installed)
%% <path to your LaTeX distribution>/doc/latex/tuda-ci
%% ============================================================================
%%
% !TeX program = lualatex
%%
%% In case you need PDF/A
% \DocumentMetadata{
% pdfstandard=a-2b,
% pdfversion=1.7,% 2.0 is possible as well, but PDF/A-2b requires < 2.0
% lang=en,
%}

\documentclass[
	english,% Main language as global option
	accentcolor=9c,% Choose accent color: For a list of available colors see the full tudapub documentation
	points=true, to activate referencing task properties
%	accept-missing-logoes=true,% No error in case logo files are not available
%	logofile=example-image,% In case logo should be replaced
]{tudaexercise}

%%%%%%%%%%%%%%%%%%%
% Language setup
%%%%%%%%%%%%%%%%%%%
\usepackage[english]{babel}
\usepackage[autostyle]{csquotes}% \enquote, to simplify use of quotation marks

%%%%%%%%%%%%%%%%%%%
% Bibliography
%%%%%%%%%%%%%%%%%%%
\usepackage{biblatex}
\addbibresource{DEMO-TUDaBibliography.bib}% File name of BibTeX database

%%%%%%%%%%%%%%%%%%%
% Document specific setup for demonstration, generally not needed!
%%%%%%%%%%%%%%%%%%%
\usepackage{booktabs}% Improved horizontal rules for use in tables
\renewcommand*{\creditformatsum}[1]{\creditformat{Total: #1}}

\ConfigureHeadline{
	headline={title-name-id}
}

\newcommand*{\file}[1]{\texttt{#1}}
\newcommand*{\code}[1]{\texttt{#1}}
\newcommand*{\pkg}[1]{\textsf{#1}}
\newcommand*{\cls}[1]{\textsf{#1}}
\newcommand*{\option}[1]{\texttt{#1}}
\newcommand*{\environment}[1]{\texttt{#1}}
\let\tbs\textbackslash
%%%%%%%%%%%%%%%%%%%
% End of demo specific setup
%%%%%%%%%%%%%%%%%%%

\title[TUDaExercise]{Tasksheet to practice use of TUDaExercise class}
\author{Marei Peischl}
\term{Spring 2042}
\sheetnumber{5}

\begin{document}

\maketitle

\begin{task}{Title}
	TUDaExercise does not support the \code{titlepage} option.
	To save space the setting is enforced to \code{titlepage=false}.
	The title material supports some additional commands, a full example is shown below:

\begin{verbatim}
\title[Short title to be used in running headers]{Title}
\subtitle{Subtitle}
\author{Author/Professor}
\term{Term}
\date{Date}
\sheetnumber{Sheetnumber}
\end{verbatim}

	All fields but the title may be kept empty.
	In case no sheetnumber is used the tasks will be numbered starting at 1.
\end{task}

\begin{task}{Coloring}
	The coloring uses the same mechanisms as the other classes of the TUDa-CI bundle. The options \option{color},
	\option{accentcolor}, \option{textaccentcolor}, \option{identbarcolor} work as usual \cite[see][]{tuda-ci}. It is also possible to
	deactivate the colored title block.
	The option \option{colorback=false} makes it possible to set the toner-saving variant for
	versions for printing
\end{task}

\begin{task}{Create a task}

	Tasks are defined in \cls{tudaexercise} using the \environment{task} environment. The required argument is used as a title. It may remain empty.

\begin{verbatim}
  \begin{task}[<options>]{title}
    Text describing the task
  \end{task}
\end{verbatim}

	Cross-references between the tasks are possible as usual (\code{\tbs{}label} and \code{\tbs{}ref}).

\end{task}


\begin{task}{Subtasks}
	\begin{subtask}[title={Basic structure}]
		\cls{tudaexercise} provides the \code{subtask} environment to subdivide tasks into subtasks.
		Currently inly one level is supported.
		For further subdivision, it is also possible to divide the document into sections using \code{\tbs{}section}.

\begin{verbatim}
\begin{subtask}[title={title},points=5]

\end{subtask}
\end{verbatim}

		In contrast to tasks, subtasks do not require a heading.
		Therefore, the syntax for the transfer differs from that of the task environment. However, there is a variant to avoid this distinction, see task \ref{task-subtask-syntax}.
	\end{subtask}

	\begin{subtask}[title={Layout modification of the headings}]
		From version 3.13, \cls{tudaexercise} also supports switching the layout of subtask headings.
		There are three option values for the document class:
		subtask=ruled
		subtask=plain
		subtask=runin If very small tasks are set, the break after the title can be prevented . In this case, the user is responsible
		for ensuring that any necessary visual separation between the title and task text takes place, e.g. by
		highlighting the title text
		Ab Version 3.13 unterstützt \cls{tudaexercise} zusätzlich eine Umschaltung bezüglich des Layouts der Subtask-Überschriften. Es existieren drei Optionswerte für die Dokumentenklasse:
		\begin{description}
			\item[subtask=ruled] Default setting, the headings are separated from the body text by lines (As shown in this document).
			\item[subtask=plain] No rules around the headings
			\item[subtask=runin] If very small tasks are set, the break after the title can be prevented . In this case, the user is responsible
			      for ensuring that any necessary visual separation between the title and task text takes place, e.g. by highlighting the title text.
		\end{description}
	\end{subtask}
\end{task}


\begin{task}{Headline customization}
	The mechanism for the headers has been adapted so that its content can be freely selected.

\begin{verbatim}
\ConfigureHeadline{
headline={content of the headline}
}
\end{verbatim}

	There is now a distinction between left and right pages (in case \option{twoside=true}).
\begin{verbatim}
\ConfigureHeadline{
even={content of the header for left pages},
odd={content of the header for right pages},
oneside={content of the header in single-page mode}
}
\end{verbatim}

	Settings that affect the same pages overwrite each other.

	\begin{subtask}[title={Custom content for headlines}]
		The content can be any \LaTeX{} code which will be placed within a box as wide as the text.
		Content will be typeset left aligned.
		It is possible to place graphics or tables within this box.

		There are some ready-made elements for the headers that can be placed in the header.
\begin{verbatim}
\ShortTitle
\StudentID
\StudentName
\end{verbatim}
	\end{subtask}

	\begin{subtask}[title={Predefined header options}]
		In addition, the \option{headline} option acceps some pre-defined values which will place the entries entries according to the self-explanatory naming scheme:
		\option{title-name-id}, \option{title-name}, \option{title}, \option{name-id}, \option{name}
	\end{subtask}

	\begin{subtask}[title={Enable the header for the title page}]
		As the header may be used to specify student data, it is possible to use the class option \option{headontitle}. This will enable the header for the title page.
	\end{subtask}
\end{task}

\begin{task}[solution=true]{Solutions}
	\cls{tudaexercise} has a mechanism for managing solution proposals within the files.
	The \environment{solution} environment exists for this purpose.
	To enable the output of solutions it's possible to use the class option (global)

\begin{verbatim}
\documentclass[...,
   solution=true,
]{tudaexercise}
\end{verbatim}
	or configure it locally for individual tasks:

\begin{verbatim}
\begin{task}[solution=true]{Heading}
Task description
   \begin{solution}
   Solution
   \end{solution}
\end{task}
\end{verbatim}

	The default setting is solution=false.

	\begin{solution}
		This is an example for a solution block.
	\end{solution}

	There also exists a \environment{solution*} environment. Which behaves the same way, but does not automatically add a title to the block.
\end{task}

\begin{task}{Query the \option{solution} setting}

	As of version 3.04, it is also possible to query the solution setting using the following structures:

\begin{verbatim}
\IfSolutionT{<if solution=true>}
\IfSolutionF{<if solution=false>}
\IfSolutionTF{<if solution=true>}{<if solution=false>}
\end{verbatim}
	In the current setting (solution=false), the above block generates the following output:

	\IfSolutionT{<if solution=true>}
	\IfSolutionF{<if solution=false>}
	\IfSolutionTF{<if solution=true>}{<if solution=false>}

\end{task}

\begin{task}[credit=10]{Points}
	Points can be specified using the \option{credit} key.

	using the macro \creditformat option key.
	The value is placed after the task title within parentheses.
	This is internally done using the \code{\tbs{}creditformat} command which can be redefined to archieve different formats.

	Alternatively, there is the \option{points} key which, in addition to the specification, also adds the text \code{\tbs{}PointName} or
	\code{\tbs{PointsName}} depending on whether the value equals 1. The points key therefore only accepts numerical
	values (from version 3.13 floating point numbers are supported).
	The decimal separator here is a period.
	Correction to a comma is possible by redefining \code{\tbs{}pointformat}.

	\begin{subtask}[points=1,title={Example for the using \option{points}}]
		As older versions of TUDaExercise did not allow option processing for subtask, an additional mechanism is
		included here.
		If both a title and specific information are required, the title key is also included here.
		A check is made to see whether the argument contains an equals sign; if this is the case, it is assumed that options are passed according to the key=value structure.
	\end{subtask}
\end{task}

\begin{task}[points=auto]{Automatisms for grading tables}
	\label{task:refPoints}

	At version 3.13 \cls{tudaexercise} received the functionality for automatic grading tables and references to the point data.

	\begin{subtask}[points=3,title={Activate references}]
		\cls{tudaexercise} has generally deactivated this functionality, but it can be activated via the class option \option{points=true} be switched on.

		The referencing mechanism works in a similar way to the object references in \LaTeX.
		This makes it necessary to compile the document twice.
		If \cls{tudaexercise} detects different values between begin and end of the compilation process, a corresponding warning is generated.

		Two macros are available for the reference.
		One works by calling up the task number, the other via labels:

		\verb+\getPoints{+\thetask\verb+}+\\
		\verb+\refPoints{task:refPoints}% requires the \label{task:refPoints}, see source code+

		In this case, both macros return the same value: \refPoints{task:refPoints}.
		As {\tbs{}refPoints} works independently of hte structure of the task numbers, this variant is generally preferable.
		In addition to references to individual tasks, \verb+\getPointsTotal+ returns the total of all tasks of the last compilation.
	\end{subtask}

	\begin{subtask}[points=2,title={Summation among subtasks}]
		The new mechanism also allows automatic calculation of the total points of all subtasks within a task.
		To enable this mechanism the option \option{points=auto} has to be set.
		If it's set as a class option, it will be activated for all tasks.
		If the class option points=auto is set, this is automatically activated for all tasks.

		However, the automatic value can be overwritten by manually entering a value, e.\,g. if a task does not onldy consist of subtasks.

		There is currently no automatic check that the sum of the subtasks matches the total.

		There is another internal variant of the macro \code{\tbs{}creditformat} to allow a distinction between summed values
		and manually set values.

\begin{verbatim}
\newcommand*{\creditformatsum}[1]{\creditformat{#1}}
\end{verbatim}

		By default it is the same but could be overwritten like the following example:

\begin{verbatim}
\renewcommand*{\creditformatsum}[1]{\creditformat{Total: #1}}
\end{verbatim}
	\end{subtask}

	\begin{subtask}[points=3,title={Automatic grading tables}]
		\cls{tudaexercise} also provides a list of all points with the option of referencing the task values.
		As the structure of a grading table varies greatly depending on personal preferences, the macro is designed in a very
		abstract way.

\begin{verbatim}
\mapPoints{<code to be executed for each task>}
\end{verbatim}

		\code{\tbs{}mapPoints} iterates over all tasks and allows any code elements as arguments.
		Within the code, \code{\#1} (Task Number) and \code{\#2} (associated point value) can be used.
		Tasks with a value of zero are skipped by default.

		For example:
\begin{verbatim}
\begin{tabular}{@{}lr@{}}
  \toprule
  \mapPoints{Task #1&#2 Points\\}
  \midrule
  Total&\getPointsTotal{} Points\\
  \bottomrule
\end{tabular}
\end{verbatim}
		creates the following output:\\
		\begin{tabular}{@{}lr@{}}
			\toprule
			\mapPoints{Task #1&#2 Points\\}
			\midrule
			Total&\getPointsTotal{} Points\\
			\bottomrule
		\end{tabular}

	\end{subtask}

	\begin{subtask}[points=5,title={Detailed grading tables including subtasks}]
		With version 4.00 there is also the option using \verb+\mapPoints*+.
		This variant also iterates over subtasks.
		In addition, the mechanism has been expanded to include a more general variant.
		This can be used to interact with all tasks.
		The documentation will be extended!
	\end{subtask}
\end{task}

\begin{task*}{Uniform syntax for task/subtask}
	\label{task-subtask-syntax}
	By default, TUDaExercise distinguishes between task and subtask. This is due to the fact that subtask does
	not require a title to be specified.

	The need for a uniform syntax was expressed (see \url{https://github.com/tudace/tuda_latex_templates/issues/189}).

	\begin{subtask*}{Starred variants of the task/subtaks environments}
		The task* and subtask* variants implemented with version 3.0 make this possible:

\begin{verbatim}
\begin{task*}{Task heading}
\begin{subtask*}{subtask heading}
Description
  \end{subtask*}
\end{task*}
\end{verbatim}
	\end{subtask*}

	\begin{subtask*}{Swapping both variants}
		The class option \code{match-task-subtask-syntax} allows subtask and subtask* to be swapped. The task* and task environments are identical anyway.
	\end{subtask*}
\end{task*}


\end{document}
%% End of file `DEMO-TUDaExercise.tex'.
