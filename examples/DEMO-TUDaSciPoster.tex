%% This is file `DEMO-TUDaSciPoster.tex' version 3.42-dev (2025-01-07),
%% it is part of
%% TUDa-CI -- Corporate Design for TU Darmstadt
%% ----------------------------------------------------------------------------
%%
%% Copyright (C) 2018--2025 by Marei Peischl <marei@peitex.de>
%%
%% ============================================================================
%% This work may be distributed and/or modified under the
%% conditions of the LaTeX Project Public License, either version 1.3c
%% of this license or (at your option) any later version.
%% The latest version of this license is in
%% http://www.latex-project.org/lppl.txt
%% and version 1.3c or later is part of all distributions of LaTeX
%% version 2008/05/04 or later.
%%
%% This work has the LPPL maintenance status `maintained'.
%%
%% The current maintainer of this work is
%%   Marei Peischl <tuda-ci@peitex.de>
%%
%% The development respository can be found at
%% https://github.com/tudace/tuda_latex_templates
%% Please use the issue tracker for feedback!
%%
%% If you need a compiled version of this document, have a look at
%% http://mirror.ctan.org/macros/latex/contrib/tuda-ci/doc
%% or at the documentation directory of this package (if installed)
%% <path to your LaTeX distribution>/doc/latex/tuda-ci
%% ============================================================================
%%
% !TeX program = lualatex
%%
%% In case you need PDF/A
% \DocumentMetadata{
% pdfstandard=a-2b,
% pdfversion=1.7,% 2.0 is possible as well, but PDF/A-2b requires < 2.0
% lang=en,
%}

\documentclass[
	english,% Main language as global option
	accentcolor=9c,% Choose accent color: For a list of available colors see the full tudapub documentation
%	boxstyle=boxed,% Round corners and colored title
%	boxstyle=colored,% Colored title without vertical lines
%	boxstyle=default,% Default: without color and vertical lines
%	logofile=example-image,% In case logo files are not available
]{tudasciposter}

%%%%%%%%%%%%%%%%%%%
% Language setup
%%%%%%%%%%%%%%%%%%%
\usepackage[english]{babel}
\usepackage[autostyle]{csquotes}% \enquote, to simplify use of quotation marks

%%%%%%%%%%%%%%%%%%%
% Bibliography
%%%%%%%%%%%%%%%%%%%
\usepackage{biblatex}
\addbibresource{DEMO-TUDaBibliography.bib}% File name of BibTeX database

%%%%%%%%%%%%%%%%%%%
% Document specific setup for demonstration, generally not needed!
%%%%%%%%%%%%%%%%%%%
\newcommand*{\file}[1]{\texttt{#1}}
\newcommand*{\code}[1]{\texttt{#1}}
\newcommand*{\pkg}[1]{\textsf{#1}}
\newcommand*{\cls}[1]{\textsf{#1}}
\newcommand{\tbs}{\textbackslash}
\newcommand{\repl}[1]{<\textit{#1}>}
\newcommand*{\macro}[1]{\code{\tbs#1}}
%%%%%%%%%%%%%%%%%%%
% End of demo specific setup
%%%%%%%%%%%%%%%%%%%
\begin{document}

\title{\pkg{tcolorbox}-Poster using TU Darmstadt's Corporate Design}
\author{Marei Peischl\inst{*}\thanks{kontakt@peitex.de} \and der \TeX-Löwe}
\institute{\inst{*}pei\TeX{} \TeX{}nical Solutions, Hamburg}
% The author and institute can use \inst for an assignment. Be careful when numbering with \thanks.
\titlegraphic{\includegraphics[width=.5\linewidth]{example-image}}
\footerqrcode{https://www.peitex.de}
\footer{For further information besides the logos}

% Institute/sponsor logos (left to right)
\footergraphics{
	\includegraphics[height=\height]{example-image}
	\includegraphics[height=\height]{example-image}
	\includegraphics[height=\height]{example-image}
}

\begin{tcbposter}[
	poster={
		columns=4,
		rows=7,
		spacing=1cm,
%		showframe,% Helpful for placements
	},
]

\begin{posterboxenv}[title=Intro]{name=intro,column=1,row=1,span=4}
	The document class \cls{tudasciposter} is based on the \pkg{tcolorbox} package by Thomas F. Sturm.
	This document is an example for an academic poster using the Corporate Design of TU Darmstadt.

	Different box types are shown in this document. They are only for demonstration.
\end{posterboxenv}

\begin{posterboxenv}[title=Title]{name=title,row=2,span=2,rowspan=2}
	There are macros like \macro{title}, \macro{author}, \macro{institute} and \macro{titlegraphic} for the title block.
	\macro{titlegraphic} is right-aligned under the TUDa logo and
	\macro{linewidth} is the width of the TUDa logo.

	There are also \macro{setqrcode} and \macro{setfoot} for the footer. An example of this is in \file{DEMO"=TUDaSciPoster.tex}.

	This is then passed with the standard macro \macro{maketitle}.
\end{posterboxenv}

\begin{posterboxenv}[title=Footer]{name=footer,below=title,span=2,rowspan=2}
	The footer can be switched off with \code{footer=false} but then additional title data with \macro{thanks} will not be displayed. Basically \code{footer=true}.

	Further macros are \macro{footer}, \macro{footergraphics} for logos and \macro{footerqrcode} that can specify a URL, which is then placed in the bottom right corner as a QRCode.

	The footer arises from \macro{thanks}. It has the width of the type area minus the logos/QR code.
\end{posterboxenv}

\begin{posterboxenv}[title=Placement of boxes]{name=positioning,below=footer,span=2}
	It is an additional step, but the boxes are positioned manually in the \pkg{poster} library of the \pkg{tcolorbox} package.
	This not only allows a nicer alignment of the boxes, but also simplifies the positioning of cross-references.
	See also \pkg{tcolorbox} documentation.
\end{posterboxenv}

\begin{posterboxenv}[title=TUDa-boxed style,TUDa-boxed]{name=boxed,column=3,row=2,span=2}
	Boxes are preset according to official specifications but sometimes it is necessary to design these individually. This can be defined with the global class option \code{boxstyle=boxed} or the local style \code{TUDa-boxed}.
\end{posterboxenv}

\begin{posterboxenv}[title=Colored box,TUDa-colored]{name=colored,column=3,row=3,span=2}
	This colored box can be defined with the global class option \code{style=colored} or the local style \code{TUDa-colored}.
	It is very similar to \code{boxed} and \code{official}.
\end{posterboxenv}

% Box with an image without without title
\begin{posterboxenv}[TUDa-colored]{name=colored-notitle,column=3,row=4,rowspan=2}
	\includegraphics[width=\linewidth]{example-image}
	\captionof{figure}{
		An example image in a box without title (\code{boxed-notitle}) where \code{TUDa} style and TUDa-colored are identical.
	}
\end{posterboxenv}

% Box without title
\begin{posterboxenv}[TUDa-boxed]{name=notitle,column=4,below=colored}
	An example of the \code{boxed} style without title.
\end{posterboxenv}

\begin{posterboxenv}[title=Box with reference,TUDa-boxed]{name=verweis,column=4,above=row6}
	An example box with an arrow that links two boxes together or generates a read branch.
\end{posterboxenv}

% TikZ code can be entered directly between the boxes. The scheme of the boxes as coordinates is
% TCBPOSTER@<boxname>.anchor point
% For more information, see tikz instructions. Further named coordinates can be found in the tcolorbox documentation.
\draw[accentcolor,line width=4pt,->] ([yshift=-1cm]TCBPOSTER@verweis.east) -|  ([xshift=1cm]TCBPOSTER@colored.east) -- (TCBPOSTER@colored.east);

\begin{posterboxenv}[title=Relative positioning,TUDa-boxed]{name=relative,column=4,between=notitle and verweis}
	This box is placed between \code{notitle} box and \code{verweis} box
\end{posterboxenv}

\begin{posterboxenv}[title=Paper size]{name=paper,column=3,span=2,below=row5}
	The paper format can be determined using the class option \code{paper}. The formats in \cls{tudasciposter} are A0, A1, A2 and A3.
\begin{verbatim}
paper=a0
\end{verbatim}
	Default is \code{a0}.

	The orientation can also be defined using the \code{paper} option.
	Landscape format is created with \code{paper=landscape}.

	If a larger design is to be changed to a smaller one, it is recommended to scale when printing. Otherwise, the PDF file can be converted using packages like \pkg{pdfpages}.
	However, it should be noted that changing the paper format does not represent scaling.
\end{posterboxenv}

\end{tcbposter}


\end{document}
%% End of file `DEMO-TUDaSciPoster.tex'.
