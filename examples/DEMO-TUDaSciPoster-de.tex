%% This is file `DEMO-TUDaSciPoster-de.tex' version 3.42-dev (2025-01-07),
%% it is part of
%% TUDa-CI -- Corporate Design for TU Darmstadt
%% ----------------------------------------------------------------------------
%%
%% Copyright (C) 2018--2025 by Marei Peischl <marei@peitex.de>
%%
%% ============================================================================
%% This work may be distributed and/or modified under the
%% conditions of the LaTeX Project Public License, either version 1.3c
%% of this license or (at your option) any later version.
%% The latest version of this license is in
%% http://www.latex-project.org/lppl.txt
%% and version 1.3c or later is part of all distributions of LaTeX
%% version 2008/05/04 or later.
%%
%% This work has the LPPL maintenance status `maintained'.
%%
%% The current maintainer of this work is
%%   Marei Peischl <tuda-ci@peitex.de>
%%
%% The development respository can be found at
%% https://github.com/tudace/tuda_latex_templates
%% Please use the issue tracker for feedback!
%%
%% If you need a compiled version of this document, have a look at
%% http://mirror.ctan.org/macros/latex/contrib/tuda-ci/doc
%% or at the documentation directory of this package (if installed)
%% <path to your LaTeX distribution>/doc/latex/tuda-ci
%% ============================================================================
%%
% !TeX program = lualatex
%%

\documentclass[
	german,% Hauptsprache als globale Option, früher war ngerman notwendig
	accentcolor=9c,% Auswahl der Akzentfarbe
%	boxstyle=boxed,% Boxen mit abgerundeten Ecken, farbigem Titelblock
%	boxstyle=colored,% Boxen mit farbigen Titelblock, keine vertikalen Linien
%	boxstyle=default,% Voreinstellung: ohne Farbe, ohne vertikale Linien
%	logofile=example-image,% Falls die Logo Dateien nicht vorliegen
]{tudasciposter}

%%%%%%%%%%%%%%%%%%%
% Spracheinstellungen
%%%%%%%%%%%%%%%%%%%
\usepackage[german]{babel}
\usepackage[autostyle]{csquotes}% Sprachabhängige Anführungszeichen mit \enquote

%%%%%%%%%%%%%%%%%%%
% Literaturverzeichnis
%%%%%%%%%%%%%%%%%%%
\usepackage{biblatex}
\addbibresource{DEMO-TUDaBibliography.bib}% Dateiname der .bib-datei

%%%%%%%%%%%%%%%%%%%
% Formatierungen für Beispiele in diesem Dokument. Im Allgemeinen nicht notwendig!
%%%%%%%%%%%%%%%%%%%
\newcommand*{\file}[1]{\texttt{#1}}
\newcommand*{\code}[1]{\texttt{#1}}
\newcommand*{\pkg}[1]{\textsf{#1}}
\newcommand*{\cls}[1]{\textsf{#1}}
\newcommand{\tbs}{\textbackslash}
\newcommand{\repl}[1]{<\textit{#1}>}
\newcommand*{\macro}[1]{\code{\tbs#1}}
%%%%%%%%%%%%%%%%%%%
% Ende der Demo-Formatierungseinstellungen
%%%%%%%%%%%%%%%%%%%
\begin{document}

\title{\pkg{tcolorbox}-Poster im Corporate Design der TU~Darmstadt}
\author{Marei Peischl\inst{*}\thanks{kontakt@peitex.de} \and der \TeX-Löwe}
\institute{\inst{*}pei\TeX{} \TeX{}nical Solutions, Hamburg}
% \inst kann in den Autor und Institutsfeldern genutzt werden, um eine Zuordnung zu ermöglichen. Bei Nummerierung ist der Nutzer dafür verantwortlich Konflikte mit \thanks zu vermeiden.
\titlegraphic{\includegraphics[width=.5\linewidth]{example-image}}
\footerqrcode{https://www.peitex.de}
\footer{Fußzeile: Falls neben den Logos andere Angaben erforderlich sind}

% Instituts/Sponsorenlogos von links nach rechts
\footergraphics{
	\includegraphics[height=\height]{example-image}
	\includegraphics[height=\height]{example-image}
	\includegraphics[height=\height]{example-image}
}

\begin{tcbposter}[
	poster={
		columns=4,
		rows=7,
		spacing=1cm,
%		showframe,% Gitter einblenden. Für Platzierung häufig hilfreich
	},]

\begin{posterboxenv}[title=Zusammenfassung]{name=intro,column=1,row=1,span=4}
	Die \cls{tudasciposter}-Klasse basiert auf dem \pkg{tcolorbox} Paket von Thomas F. Sturm.
	Sie versucht einen einfachen Weg zu bieten, wissenschaftliche Poster im Corporate Design der TU Darmstadt zu erstellen. Dieses Dokument dient zur Dokumentation und als Verwendungsbeispiel.

	Dieses Dokument verwendet unterschiedliche Boxentypen. Dies ist selbstverständlich für die praktische Verwendung nicht empfehlenswert. Dieser Modus dient lediglich Demonstrationszwecken.
\end{posterboxenv}

\begin{posterboxenv}[title=Titelei]{name=title,row=2,span=2,rowspan=2}
	Die Definition des Titelblockes lehnt sich an die Standard-\LaTeX{}-Strukturierung mit Hilfe von \macro{maketitle} an.

	Für die Datenübergabe stehen die Makros \macro{title}, \macro{author}, \macro{institute} und \macro{titlegraphic} zur Verfügung. Letztere wird rechtsbündig unterhalb des TUDa-Logos platziert. Der Wert von \macro{linewidth} zu diesem Zeitpunkt entspricht der Breite des TUDa-Logos.

	Zusätzlich zu den Titeldaten stehen über \macro{setqrcode} und \macro{setfoot} Makros zur Verfügung, die die Fußzeile füllen.
	Ein Beispiel ist in der Datei \file{DEMO"=TUDaSciPoster.tex} gezeigt.
\end{posterboxenv}

\begin{posterboxenv}[title=Fußzeile]{name=footer,below=title,span=2,rowspan=2}
	Die Fußzeile ist grundsätzlich aktiviert, kann jedoch mit Hilfe der Klassenoption \code{footer=false} ausgeschaltet werden. In diesem Fall werden jedoch mit \macro{thanks} übergebene zusätzliche Titeldaten nicht angezeigt.

	Für die Übergabe weiterer Daten stehen die Makros \macro{footer}, \macro{footergraphics} und \macro{footerqrcode} zur Verfügung.

	\macro{footergraphics} ist für die Übergabe von Logos gedacht und \macro{footerqrcode} übernimmt einen URL die anschließend in der rechten unteren Ecke als QRCode platziert wird.

	Die Fußzeile selbst erhält die Daten aus \macro{thanks}, kann jedoch ergänzt werden. Sie hat die Breite des Satzspiegels abzüglich der Logos/QRcode.
\end{posterboxenv}

\begin{posterboxenv}[title=Platzierung der Boxen]{name=positioning,below=footer,span=2}
	Bei der \pkg{poster}-Bibliothek des \pkg{tcolorbox} Paketes, werden die Boxen manuell positioniert.

	Dies benötigt zwar einen zusätzlichen Arbeitsschritt, erlaubt jedoch einer feinere Ausrichtung der Boxen, auch relativ zueinander.

	Diese Mechanismen ermöglichen auch, Querverweise einfacher zu positionieren. Hierfür ist ein Blick in die \pkg{tcolorbox}-Dokumentation hilfreich.
\end{posterboxenv}

\begin{posterboxenv}[title=Eine Box im Stil TUDa-boxed,TUDa-boxed]{name=boxed,column=3,row=2,span=2}
	Die Boxen können in verschiedenen Varianten gestaltet werden. Die Voreinstellung entspricht den offiziellen Vorgaben, jedoch kann es aus unterschiedlichen Gründen notwendig sein, eine klarere Abgrenzung zu setzen (globale Klassenoption \code{boxstyle=boxed} oder lokaler Stil \code{TUDa-boxed}).
\end{posterboxenv}

\begin{posterboxenv}[title=Eine Box im Stil TUDa-colored,TUDa-colored]{name=colored,column=3,row=3,span=2}
	Boxentyp zwischen dem Stil \code{boxed} und dem Stil \code{official}.

	Einstellung dieser Option ist sowohl über die Nutzung der globalen Klassenoption \code{style=colored} als auch durch die Verwendung des lokalen Stils\code{TUDa-colored} möglich.
\end{posterboxenv}

% Box ohne Titel mit Abbildung
\begin{posterboxenv}[TUDa-colored]{name=colored-notitle,column=3,row=4,rowspan=2}
	\includegraphics[width=\linewidth]{example-image}
	\captionof{figure}{
		Ein Beispielbild in einer Box ohne Titel (\code{boxed-notitle}). In diesem Fall sind der Stil \code{TUDa} und TUDa-colored identisch.
	}
\end{posterboxenv}

% Box ohne Titel
\begin{posterboxenv}[TUDa-boxed]{name=notitle,column=4,below=colored}
	Ein Beispiel für den Stil \code{boxed} ohne einen eigenen Titel.
\end{posterboxenv}

\begin{posterboxenv}[title=Box mit Verweis,TUDa-boxed]{name=verweis,column=4,above=row6}
	Beispielbox mit Pfeil, um zwei Boxen miteinander zu verknüpfen oder Leseabzweigungen zu generieren.
\end{posterboxenv}

% Zwischen den Boxen kann direkt TikZ-Code eingegeben werden. Das Namensschema der Boxen als Koordinaten lautet
% TCBPOSTER@<boxname>.ankerpunkt
% Für genauere Erläuterungen zur Syntax, bietet die tikz-Anleitung genauere Angaben. Weitere benannte Koordinaten finden sich in der tcolorbox-Dokumentation.
\draw[accentcolor,line width=4pt,->] ([yshift=-1cm]TCBPOSTER@verweis.east) -|  ([xshift=1cm]TCBPOSTER@colored.east) -- (TCBPOSTER@colored.east);

\begin{posterboxenv}[title=Relative Positionierung,TUDa-boxed]{name=relative,column=4,between=notitle and verweis}
	Diese Box wird zwischen den beiden Boxen \code{notitle} und \code{verweis} platziert.
\end{posterboxenv}

\begin{posterboxenv}[title=Papierformat]{name=paper,column=3,span=2,below=row5}
	Die Klasse \cls{tudasciposter} unterstützt die Papierformate A0, A1, A2 und A3. Der Wert wird über die Klassenoption \code{paper} ausgewählt:
\begin{verbatim}
paper=a0
\end{verbatim}
	Die Voreinstellung entspricht \code{a0}.
	Die Änderung des Papierformates ist keine Skalierung, da Schriftgrößen nicht direkt skalieren.

	Zusätzlich kann die Option \code{paper} wie bei \KOMAScript{} üblich auch einen Wert für die Ausrichtung verarbeiten. Querformat erhält man somit über \code{paper=landscape}.

	Um eine Skalierung eines größeren auf ein kleineres Design zu erreichen, empfiehlt es sich, das Ausgangsformat beim Druck zu skalieren (Drucken in eine Datei mit Skalierung) oder ggf. die PDF-Datei mit Paketen wie \pkg{pdfpages} umzurechnen.
\end{posterboxenv}

\end{tcbposter}


\end{document}
%% End of file `DEMO-TUDaSciPoster-de.tex'.
